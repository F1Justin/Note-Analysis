\documentclass{ctexart}
\usepackage{geometry}
\usepackage{fontspec}
\usepackage{tikz}
\usepackage{hyperref}  
\usepackage{CJKfntef}
\usepackage{tabularx}

\hypersetup{hidelinks,
	colorlinks=true,
	allcolors=black,
	pdfstartview=Fit,
	breaklinks=true
}

\newcommand{\mybox}[1]{\tikz[baseline=(MeNode.base)]{\node[rounded corners, fill=gray!20](MeNode){#1};}}
\geometry{papersize={21cm,29.7cm}}
\geometry{left = 2.7cm, right = 2.65cm, top = 2cm, bottom = 3cm}
\pagestyle{headings}
\setCJKmonofont{LXGW WenKai Mono}
\ctexset {
   abstractname = {本文概要},
   today = big,
   section/name = {第,节},
}

\newcommand{\df}{\large \mybox}
\newcommand{\nm}{\normalsize}
\newcommand{\blk}{\vspace*{1\baselineskip} }
\newcommand{\blkz}{\vspace*{2\baselineskip} }
\newcommand{\blkx}{\vspace*{4\baselineskip} }
\newcommand{\blkc}{\vspace*{6\baselineskip} }
\newcommand{\blkd}{\vspace*{10\baselineskip} }
\newcommand{\blkv}{\vspace*{16\baselineskip} }
\newcommand{\blkb}{\vspace*{32\baselineskip} }
\renewcommand{\\}{\par}

\setcounter{secnumdepth}{4}
\setcounter{tocdepth}{2}

\begin{document}


\title{语文分析}
\author{F1}
\maketitle
\linespread{1.48}

\tableofcontents
\newpage
%『』
\part{答题模板}

\section{叙述,描写与抒情}

\subsection{『比喻』修辞之比喻}
比喻是常用的修辞手法.\\
\df{比喻/比拟?}\\
\df{描绘了什么样的画面?}\\
\df{能怎样表达主旨?}\\
\subsubsection{例题}
请分析第二段划线句在文中的作用.\\
\texttt{句中将人,云,山分别比作浮萍,睡莲,微生物,吐出了人与自然事物在宇宙背景下的渺小,比喻大胆新奇,吸引读者阅读兴趣.为后文描写雨,云,水做铺垫,奠定圣洁高远的基调,反映了作者的崇敬之情.}
\subsubsection{例题}
第三段划线句很有表现力,请加以赏析.\\
\texttt{本句运用比拟手法,赋予水银以人的情态,"欢欣鼓舞","朝我眨眼",生动表现出水银泄地后程珠状扩散之快与我在黑夜中见此情景的以外之喜.又运用比喻手法,将泄地的水银比作满天繁星,不仅生动渲染了银珠之多,还巧妙地将天地空间倒置,与下文的时间回溯和谐地融为一体.}\\
\nm \fangsong
\centerline{风吹云动}\\
\centerline{傅菲}\\
\\天空一无所有
\\何以给我安慰
\\这是海子的诗句。其实天空有云。云也只游荡在天空里。天空是云的居所。
\\云可能是最轻的东西了,它终生被风吹动。风托着它,拽着它,改变它的形状。风让云聚成一团,也让云成流丝。山区多云,也多风。荣华山的上空,盘踞着云,满池塘浮萍似的,让人卑微:人只是池塘里的微生物,荣华山也只是一朵水莲。
\\荣华山草木葱茏,水蒸发量大,多云是惯常的。云带来了雨。或者说,云是雨的前世,雨是云的凡胎。凡胎注定在大地上浪迹。
\\初入荣华山,是夏季。炽热炎炎。我一下子注意到了云。云白如洗,蚕丝一样。天空蓝,蓝得没有尽头,蓝得深邃无穷。我对本地人疑惑地说:这天蓝得只剩下云的白了,过滤了一样。本地人望望天,说:云黑起来才吓人呢,像藏着恶魔。
\\四个月后,我见识了恶魔一样的云。白露没过几天,气温急剧下降。午后,天完全黑暗,山下盆地像个地窖。蚂蚁慌乱。院子里来了很多蜻蜓,四处飞。天是在十几分钟内暗下来的,空气如洇开了墨水。我关掉电闸,收拾翻晒的物什,坐在走廊。云乌黑黑,一层层压实铺开。云团山峦一样,一座连一座,形成绵绵群山。高耸陡峭。云团不移动,遮蔽了光,给人压迫感。
\\游动的光,蓝色,在云层突闪,爆出蜘蛛丝一样的裂缝。闪电来了。我们不叫闪电,叫忽显。忽然显现的光,照见了云团狰狞的面目。云团像戴着黑色面具,披头散发的傩舞人。雷声从天边轰轰轰传来,俯冲而下,隆隆隆隆,炸裂。闪电一道追一道,显得迫不及待。蓝色火焰啪啪啪瞬间熄灭。似乎它快速地到来,是为了快速地熄灭。云团被一层层炸开。
\\雨下了。豆珠一样,啪哒啪哒,急急地敲打地面,溅起干燥的灰尘。脆脆的雨声,犀利。雨珠打在白菜上,菜叶弹起来。雨点密集起来,雨线直拉拉。雨线网住了视野。鱼从水塘跃起。蝉声消失耳际。芭蕉花一朵朵打落在地。天慢慢白,把暗黑色一层层蜕下来,露出水光色。
\\杂工老钟每天出门,戴一顶旧得发黄的草帽,帽檐低低。他望望天,说:今天没有雨。或者说:云压头上了,有大雨。他把草帽当扇子,边摇边说话。他也把草帽当坐垫,草帽往屁股下一塞,摸出烟,说:这个天会不会热死人啊。汗滴在眉毛上了,抬手用衣袖擦。衣袖两边结了很多盐花。他是靠天吃饭的人。挖菜地劈木柴遮秧苗,都是他的活。起风了,我站在窗口,喊:疤脸,疤脸,看看云,会不会下雨啊。疤脸是老钟绰号。他喜滋滋地翘着烟,说:这个天下不了雨,别看云那么厚,风吹吹便没了。
\\山里的人,都会观云识天气。挖地,挖了一半,把锄头扛在肩上回家了。问他:怎么不挖完就回去了。他嘟嘟嘴巴,说,你也不看看云,暴雨马上来。云团还在天边呢!这里艳阳当空。可隔不了一碗茶时间,乌云盖顶,噼噼啪啪,暴雨来了。
\\云怎么也散不去,厚厚的,一堆叠一堆。云是最高、体量最大的山峦。山峦慢慢塌,以暴雨的方式坍塌。荣华山四周的盆地成了云山的地下河。云带来了充沛的雨量。手抓一把土,水飚射。辣椒烂根在地里。昨夜刚开的蔷薇,被无情地摧残。南浦溪的木船不知道漂到哪里去了。瓦漏了,哗哗哗的雨水落在了锅里,落在木板床上。过河的山麂溺水而死。
\\最彻底的洗礼。云再一次把大地恢复了原始的模样。摧枯拉朽是最彻底的清洗。云有一双魔手,让即将死亡的加速死亡,让无力生存的加速腐烂,让散叶开花的尽快茁壮成长。腐朽的,僵硬的,都埋到泥浆里去吧。
\\云随时随刻都有一种寄情穹宇的状态。像一个不问人间的隐居者。王维有诗《终南别业》:
\\中岁颇好道,晚家南山陲。
\\兴来每独往,胜事空自知。
\\行到水穷处,坐看云起时。
\\偶然值林叟,谈笑无还期。
\\散步到流水尽头,云正好从山头涌上来。其实流水没有尽头,尽头之处是终结之处。在乡村寿枋(棺材的别称)有一副常见的对联:水流归大海,月落不离天。万事万物,都遵守恒定律。人需要从容生活,淡定,淡雅,淡泊。王维被尊为诗佛,他了悟水云之禅。他四十开始,半官半隐,在陕西蓝田的辋川寄情山水。辋川青山逶迤,峰峦叠嶂,幽谷流瀑布,溪流潺湲。我想起自己不惑之年,仍在外奔波,让家人牵挂,多多少少有些悲伤。
\\陈眉公辑录《小窗幽记》,引用洪应明的对联:
\\宠辱不惊,闲看庭前花开花落
\\去留无意,漫随天外云卷云舒
\\云自卷自舒。又几人可以卷舒呢。人永远没有满足的时候,人永远不会珍惜已拥有的。水到了大海,才知道,哦,所有的行程只为了奔赴大海。
\\窗外,是晚霞映照的山峰。入秋的风,一天比一天凉。干燥的空气和干燥的蝉声,加深了黄昏的荒凉。夕阳的余光给大地抹了一层灰色。云白如翳。一个穿深色蓝衫的人,坐在溪边的石墩上画油画。他每天都来,坐在同一个石墩,已经有半个月了。我偶尔去看看他画画。他画田畴,画山梁,画云。云像什么,我们便会想什么。云,是心灵绽放出来的花。云是云,我们是我们。云不是云,我们不是我们。云是浮萍,我们是微生物。
\blkx
\\\centerline{水银花开的夜晚}
\\\centerline{迟子建}
\\有一日傍晚咳嗽流涕,我便取放在玄关托盘上的体温计,想看看自己是否发烧。
\\我取体温计的时候,不慎将外壳的护帽朝下,这一竖不要紧,由于对接处咬合不严,护帽叛徒似的落地而逃,将体温计彻底出卖了,它随之坠落,摔成两截。
\\它这一跌,我家的黑夜亮了。
\\从玻璃管内径流溢而出的水银,魔术般地分裂成大大小小的珍珠状颗粒,像一带雪山巍峨地屹立在我面前。我先是拿来一块抹布擦拭,以为它们会像水滴一样,迅速被吸附,岂料它们欢欣鼓舞地一分二、二分三、三分四地遍撒银珠,泻地水银非但未少,反而如满天繁星,在白桦木地板上,朝我眨眼。它们近在咫尺,却仿佛远在天边,不可征服。
\\我少时对水银的了解,竟来自当时广为流传的一本小人书《一块银元》,主要情节围绕一块银元展开,写了穷人的苦,地主的恶,其中最让人惊悚的情节,是一个地主婆死了,她的儿子竟让一对童男童女为他老娘殉葬。他们给童男童女灌注了水银。故事浓墨重彩的是那个身世凄惨的童女,在出殡的行列中,她端坐在莲花上,手持一盏纱灯,双目圆睁,虽死犹生。她的亲人在路旁声声唤她,可她无法应答了。那个画面给我幼小的心灵,带来了浓重的阴影,恨地主,也恨水银。水银是毒蛇,它要了如花似玉的姑娘的命!
\\我那时感冒了,发烧了,抗拒去卫生所,骨子里是恐惧水银体温计。总觉得我的腋窝藏着火苗,会将爆竹似的它引爆。它灿烂了,我就黑暗了。体温计是恶魔,这在看过 《一块银元》 小人书的同学心中,根深蒂固。以至于我们憎恨一位班主任老师时,私下议论要是小人书中被灌注了水银的是她,而不是那个女孩,该有多好。这位班主任是我们的语文老师,她中等个,微胖,圆脸上生满雀斑,厚眼皮,眼睛不大,但很犀利。
\\我们为什么怕这位老师呢? 她严厉起来不可理喻。她有一杆长长的教鞭,别的老师的教鞭只在黑板上跳舞,她的教鞭常打在学生手上。期中期末考试总成绩不及格者,是她惯常教训的对象。她会让他们伸出手来,这时她的教鞭就是皮鞭了,抽向落后生。痛和屈辱,让被打的同学哇哇大哭。这种示众的效果,倒是让所有的学生不甘落后,刻苦学习了。但大家心底对她还是恨的,她头发浓密,梳着两条粗短的辫子,我们背地就说她带着两把锅刷。
\\最让我们难堪的是检查个人卫生,我们上课前她会手持碎砖头,高傲地站在门口,我们则像乞丐一样朝她伸出手去,如果我们的手皴了,或是指甲里藏污纳垢,她会扔给你一块碎砖头,让我们出去蹭掉手上的皴,抠出指甲里的泥,砖头在此时就成了肥皂了。
\\这位班主任老师看上去跋扈,但她业务好,很敬业,也有善心。有的同学家贫,她家访时会带上她买的作业本,她还帮助交不起学费的学生交费,并带我们进城,去照相馆拍合影。如果是冬天,天黑得早,讲台就点起一根蜡烛。烛火跳跃着,忽明忽暗,她的脸也忽明忽暗,那也是她最美的时刻。她不用教鞭,脸上的雀斑看不见了,语气温柔,面目平和。
\\她离开我们小镇,似乎没有任何预兆。突然有一天,她要调到她恋人那儿,是去结婚。这时我们才意识到她是一个女人,是个有人惦念的人。
\\她要离开了,按理说我们是该同声庆祝的,可大家突然都很沮丧。她将自己所用之物,分给常遭她鞭打的人,那多是家庭困难的同学,我听说的就有书本、衣物、脸盆。在她走前,有天我在小卖店碰见她,她还买了一双雨靴送我。从此后她离开的风雨时刻,穿着雨靴走在泥水纵横的小路上,总会想起她。而她带我们拍的合影,成了同学们最美的珍藏。
\\四十多年了,我没有她的任何消息,也极少想起她来。但水银泄地的这个夜晚,也过了半百之岁的我,却很热切地思念起她来。不知她是否还在她当年嫁过去的小城。按她的年龄,应是儿孙满堂,颐养天年了。
\\夜一点点地黑起来,我清理完地板上的水银,关了厅里的灯,打算回卧室休息一下。借着卧室的微光,我突然发现刚清理过的地板上,仍有水银珠一闪一闪的。我不相信,取了手电筒照向那里。呵呀,这分明是一个微观花园么,我发现了无数颗更加细小的水银珠粒,在白桦木地板的表面和缝隙,花儿一样绽放着。
\\这不死的花朵,实难相送,那就索性不送,我不相信就凭它们,会让我性命堪忧———将其当花来赏又如何! 权当它们是腊梅的心,是芍药的眼,是丁香的小袄,是莲花的罗裙!
\\因为在黑夜面前,所有的花朵都是无辜的。


\subsection{『段语』``这段话在文中的作用是?''}
对于提问一段话在文中的作用的题目,要注意以下几个点:\par
\df{承接上文内容概括}, \\
\df{启接下文内容概括}, \\
\df{这段话的概括}, \\
\df{这段话的文学性}\nm (如果有的话),\\
 \df{对全文情感主旨的作用}\nm 比如:凸显.

\subsection{『风格』本诗/词的风格是...}
诗词风格多种多样, 但是如果按照抒情方式分类主要有\par
\df{委婉}\nm ---寓情于景和\par
\df{豪放}\nm ---直抒胸臆\large \par
两类.

\subsection{『群像』塑造众多人物形象的作用是...}
人物群像的描写重在由共性与个性的对比中突出\textbf{氛围}.\\
\df{共性}\\
\df{个性}\\
\df{氛围}\\
\large
\subsubsection{例题}
与聚焦单个人物形象不同,本文塑造了众多人物形象,分析这种写法的作用.\\
\texttt{本文描写了多个典型人物,如勤恳工作热于助人急性子的阿德;善于沟通引导病人的童医生;走到哪都不忘交易的淳朴的村妇们;笨拙又清闲的书贩.共同描写出乡村淳朴热闹的氛围,比起单独人物描写更有利于表现勤劳可爱的劳动人民群像,体现了作者的赞美之情.}\\
\nm \fangsong
\\\centerline{被劝进来的病人}   \\\centerline{干亚群}
\\逢三与逢七,是小镇的市日,类似于我老家的赶集。碰上市日,医院比较忙,病人把集赶了,顺带把自己的病也看了。
\\市日把村民赶到镇上。毫无遮拦的市场里,村民们挑肥拣瘦,掂斤捻两,最后以惊人的耐心杀价掐价。市场上的果蔬大多是自产自销,所以,他们买卖人的身份一个月里经常在换,轮到别人向自己砍价时,嘴上吵吵嚷嚷,手上却不让人吃亏,秤早已捏了起来,秤尾往上一翘,顾客的头随之一歪,一桩生意就完成了。
\\太阳跳上树梢,把市场照得像块煎饼时,人们各自完成买与卖,然后周围的声音慢慢浅下去,摊位上的东西也渐渐薄起来,零乱的脚印,散落的垃圾,以及花花绿绿的鸡屎,跟灵感跑了一半的油画似的。
\\市日把一撮人劝进了医院。他们带着集市的痕迹来看病。他们把拖拉机的突突声拐进了医院的大门,手推车咕噜咕噜,一个侧身依在墙角,自行车前架后搁,心事重重似的靠过来,医院的天井一点点被它们拥塞。清洁工阿德挥舞着扫帚,指挥着拖拉机停这边,手推车放那边,至于自行车,一律摆到车棚。容不得商量。一旦有人把车放错了位置,阿德就提着扫帚跑过去,如果来人不配合,阿德的脸就开始涨红,话也结巴,脖子上青筋凸起。如果有人来看病找不到医生,他会满医院地帮忙去找,一边找,一边大声咳咳,似乎在打暗号。  
\\到了医院,买卖人变成了病人。对他们而言,医院跟集市无非是换了个场景,仍用刚才吵吵嚷嚷的声音陈述自己身上的某个痛点。医生当然不会仅限于病人一句肚痛头晕就开方子,肯定要问清肚痛的来龙去脉、前因诱因。而病人翻来覆去跟烙饼似的停留在自己的痛点上,医生需要的信息仍云遮雾绕。
\\我坐在童医生对面,彼此是同事,但在和病人打交道这件事上,她是我老师。吵吵嚷嚷的声音里,童医生看上去很惬意,看见病人既不问病史,也不做检查,而是先笑嘻嘻地问病人今天市日又买了啥,然后夸病人会买东西,价格实惠。病人听了,似乎觉得自己捡了一个大便宜,语气开始亲切起来,甚至掀开篮子给童医生看自己买的东西,童医生侧过身,极认真地看了看病人的篮子,再次夸病人会买东西。俩人像是街头偶遇的老朋友拉起了家常,饮食咸淡,起居习惯,聊天把买卖人劝进病人角色。他们一股脑儿地把自己最近的生活史复习了一遍。就在病人絮絮叨叨时,医生的问话戛然而止,一张处方已递到病人面前,仿佛是市日里的一杆秤。
\\病人一坐到我前面,我根本不会像童医生那样转弯抹角地先跟病人温习市日,而是直截了当地开启病人与医生的模式。他们的病痛大多是积累起来的,问他们为什么不早点来看,回答几乎是一模一样,等市日时来看,似乎特意来看病是一件难为情的事。    
\\他们看过内科看外科,看过外科看牙科,一次次来到医生面前。而她们,闪进了右侧的诊室。她们进来时不像是看病,倒像探病,一身花衣服。她们手里提着七七八八的东西,声音也是七七八八,似乎集市的热闹仍然悬在舌头上。有时等我给病人做好检查出来时,发现突然少了几个人,原来是跑到院子里做生意去了。她买她的花裤子,她买她的红番茄,然后,俩人你提着我的花裤子,我拎着你的红番茄,再次进入诊室,脸上荡漾着番茄红。还好,她俩的病情不一样,否则我真怀疑她们刚才把病也交易了。
\\市日上的事,像边角余料似的被病人带进了医院。有人说有一个老头,每次市日摆旧书摊,可等他把书摆好,市日就散了,于是他又把书一本本收起来,几乎没有做过一笔生意,看上去像来晒书的。我置身在他们的闲谈中,忍不住问,他是卖的,还是租的?说话的人摇摇头,然后一屁股坐到童医生那儿,似乎把老人旧书摊这件事压了下去。
\\虽然市日是医院看病最忙的日子,但病人看病的时间都不长,大多病人出去时手里只不过多了一张方子,有的甚至方子都没有。到了十点半后,重新空荡荡的,却留下了一堆堆的花花绿绿,已经分不清是鸡屎盖着鸭屎,还是鸭屎压着鹅屎,唯一可以辨别的是羊粪,院长戏称是“六味地黄丸”。
\\阿德站在院子里咳咳咳。不一会儿,大家从科室里出来,脖子上挂着听诊器,而手里提着扫帚、冲水器,听从阿德的指挥,开始清扫院子,仿佛走的是客人。



\subsection{『人形』请评点此处人物形象的描写}
对于人物形象的问题,需要注意\\
\df{神态变化}\\
\df{情感变化}\\
\df{体现了什么}\\
\subsubsection{例题}
\large 
从人物描写的角度,为甲段划线句写一段评点文字\\
\texttt{划线句使用了肖像描写和语言描写,连用动词"一怔","睁开","咧开嘴"等,生动地描绘了女人见到水生后由惊愕到欣喜再到数年以来委屈的悲涌上心头的心理变化,体现了女人对水生的思念之情感之深,反映出过往战争的日子的艰难.}\\
\nm \fangsong \\\begin{center}嘱咐\end{center}\\\begin{center}孙犁\end{center}\\太阳平西的时候,水生望着树林的疏密,辨别自家的村庄,他的家就在白洋淀边上。家近了,就要进家了!他想着许多事,父亲是不是还活着?父亲很早就有痰喘病;还有自己的女人,一别八年,分别时她肚子里正有了孩子,是不是都活着?房子被烧了吗?\\他在院子门口遇见了自己的女人。她正悄悄地关闭那外面的梢门。水生叫了一声:“你!”\\女人一怔,睁开大眼睛,咧开嘴笑了笑,就转过身子抽抽打打地哭了〔甲〕。水生看见她脚上那白布封鞋,就知道父亲准是不在了。两个人愣在那里站了一会。还是水生把门掩好,说:“不要哭了,家去吧!”他在前面走,女人在后面跟, 走到院里,女人紧走两步赶在前面,到屋里去点灯。\\他走进屋里,女人从炕上拖起一个孩子来,含泪笑着说:“来!这就是你爹, 一天价看见人家有爹,自己没爹,这不回来了。”说着已经不成声音。\\水生说:“来!我抱抱。”那孩子从睡梦里醒来,好奇地看着这个生人。 水生在黑影里问:“你叫什么?”“小平。”“几岁了?”女人转身插好门,对孩子说:“别告诉他,他不记的吗?”\\水生看着女人。离别了八年,她并没有老多少,头发虽然乱,脸孔苍白了一些,可那两只眼睛里的光,还是那么强烈。\\
女人歪在炕上,笑着问:“说真的,这八九年,你想起过我吗?” “想过。”“怎么想法?”她逼着问。\\“临过平汉路的那天夜里,我宿在一家小店,小店里有个鱼贩子是咱们乡亲。我买了一包小鱼下饭,吃着那鱼,就想起了你。”\\“胡说。还有吗?”“没有了。你知道我是出门打仗去了,不是专门想你去了。”\\“我们可常常想你,黑夜白日。”她突然支着身子坐起来,问:“你能在家住几天?”\\“就这一晚上。我是请假绕道来看你的。”“为什么不早些说?”“还没顾着啊!”\\女人呆了。她低下头去,无力地仄在炕上。过了半天,她说:“那么就赶快休息吧,明天我撑着冰床子去送你。”〔乙〕
\\鸡叫三遍,女人就起来给水生做了饭吃。这是一个大雾天,地上堆满了霜雪。女人把孩子叫醒,穿得暖暖的,背上冰床,锁了梢门,送丈夫上路。出了村,她要丈夫到爹的坟上去看看。水生说以后回来再去,女人坚持要去。她说:\\“爹叫你出去打仗了,是他一个老人家照顾了全家。这是什么日子呀?整天价东逃西窜。你不在家,爹对我们娘俩的照顾,只怕一差二错,对不起在外抗日的儿子。夜里一有风声,他就把我们叫醒。他老人家背着孩子逃跑,累的痰喘咳嗽。这些个担惊受怕的日子,把他老人家累死。还有那年大饥荒……”
\\在河边,他们上了冰床。水生坐上去,抱着孩子,用大衣给她包好脚。女人站在床子后尾,撑起了竿。女人是撑冰床的好手,她逗着孩子说:“看你爹没出息,当了八年八路军,还得叫我撑冰床子送他!”\\她轻轻地跳上冰床子后尾,像一只雨后的蜻蜓爬上草叶。轻轻用竿子向后一点,冰床子前进了。大雾笼罩着水淀,只有眼前几丈远的冰道可以望见。河两岸残留的芦苇上的霜花飒飒飘落,衣服上立时变成银白色。她用一块长的黑布紧紧把头发包住,脸冻得通红,嘴里却冒着热气。她连撑几竿,然后直起身子来,向水生一笑。小小的冰床像离开了强弩的箭,摧起的冰屑,在它前面打起团团的旋花。前面有一条窄窄的水沟,水在冰缝里汹汹地流,她只说了一声“小心”,两脚轻轻地一用劲,冰床就像受了惊的小蛇一样,抬起头来,窜过去了。\\水生提醒她说:“你慢一些,疯了?”女人擦一擦脸上的冰雪和汗,笑着说: “同志!我送你到战场上去呀,你倒说慢一些!”\\“擦破了鼻子就不闹了。”“不会。这是从小玩熟了的东西,今天更不会。在这八年里面,你知道我用这床子,送过多少次八路军?”\\冰床在霜雾里飞行。“你把我送到丁家坞,”水生说,“到那里,我就可以找到队伍了。”\\女人呆望着丈夫。停了一会,才说:“你知道,我现在心里很乱。八年才见到你,你只在家呆了不到多半夜的工夫。我为什么撑的这么块?为什么着急把你送到战场上去?我是想,快快打走敌人,你才能快快地回家。”\\冰床滑进水淀中央,这里是没有边际的冰场。太阳从冰面上升起来,冲开了雾,形成一条红色的胡同,扑到这里来,照在冰床上。女人说:“爹活着的时候常说,日本人在这里,水生出去是打开一条活路,打开了这条路,我们就能活。你记着爹的话,不要为家里的事分心,好好打仗,我等你回来。”\\在杨柳树环绕的丁家坞村边,水生下了冰床。\\女人忍住泪,笑着说:“快去吧你!记着,好好打仗,快回来,我们等着你的胜利消息。”\\ \rightline{一九四六年河间}
\songti





\subsection{『视角』之感官``请鉴赏划线句的表现力''}
此类题目灵活多变,需要灵活处理.主要需要注意\\
\df{内容}\nm 修辞,分析,情感\\
\df{结构}



\subsection{『视角』之人称``本文人称的表达效果是?''}
\df{第一人称}\nm 增强带入感,引发直观真切的体验. \\
\df{第一人称儿童视角}\nm 天真细致,引发成人后的反思.\\
\df{第二人称}\nm 跳出个人视角, 隐含了与读者的对话,拉近与读者的距离,产生与读者的共情.\\
\df{第三人称}\nm 增强叙述说理的客观性,但也令读者感到疏远.
\large
\subsubsection{例题}
赏析以下选段:
\\\nm \fangsong
\\大堰河, 为了生活, 
\\在她流尽了她的乳汁之后, 
\\她就开始用抱过我的两臂劳动了; 
\\她含着笑,洗着我们的衣服, 
\\她含着笑,提着菜篮到村边的结冰的池塘去,
\\她含着笑,切着冰屑悉索的萝卜, 
\\她含着笑,用手掏着猪吃的麦糟, 
\\她含着笑,扇着炖肉的炉子的火, 
\\她含着笑,背了团箕到广场上去, 晒好那些大豆和小麦,
\\大堰河,为了生活, 
\\在她流尽了她的乳液之后, 
\\她就用抱过我的两臂,劳动了。
\large\songti
\\\texttt{本段用了第三人称,描写了大堰河为我辛苦劳作的画面,体现出了作者与大堰河之间地主儿子和贫苦农妇间的身份差异与厚障壁,反映了二人之间情感逐渐的疏远.}
\subsubsection{例题} 
赏析以下选段: 
\\\nm \fangsong
\\大堰河,今天,你的乳儿是在狱里, 
\\写着一首呈给你的赞美诗, 
\\呈给你黄土下紫色的灵魂, 
\\呈给你拥抱过我的直伸着的手, 
\\呈给你吻过我的唇, 
\\呈给你泥黑的温柔的脸颜,
\\呈给你养育了我的乳房, 
\\呈给你的儿子们,我的兄弟们, 
\\呈给大地上一切的, 
\\我的大堰河般的保姆和她们的儿子, 
\\呈给爱我如爱她自己的儿子般的大堰河。
\large\songti
\\\texttt{本段运用第三人称,表达了作者对大堰河与千千万万大堰河般勤劳善良但又命运悲苦的普通农妇的歌颂与真诚同情.反映了作者的真实情感,拉近与读者的距离,令人感动.}

\subsection{『视角』之时间``随时间推移切换场景,赏析其妙''}
我们需要注意以下三点:\\
\df{环境}\\
\df{人物}\\
\df{场景}\\
\subsubsection{例题}
从``鸡叫三遍''到结束,小说随着时间推移切换场景,赏析其构思之妙.
\texttt{开头用``寒冷黎明''交代水生父亲之死 ,体现生活艰难;女人快速划冰床送水生上战场,体现了她的识时务与支持抗战;女人对水生的殷勤寄语更是反映了希望战争快速胜利的希望;最后"太阳升起""形成了红色的胡同"象征着通向胜利之路.}
\nm \fangsong
\\\centerline{原文见上}
\\鸡叫三遍,女人就起来给水生做了饭吃。这是一个大雾天,地上堆满了霜雪。女人把孩子叫醒,穿得暖暖的,背上冰床,锁了梢门,送丈夫上路。出了村,她要丈夫到爹的坟上去看看。水生说以后回来再去,女人坚持要去。她说:\\“爹叫你出去打仗了,是他一个老人家照顾了全家。这是什么日子呀?整天价东逃西窜。你不在家,爹对我们娘俩的照顾,只怕一差二错,对不起在外抗日的儿子。夜里一有风声,他就把我们叫醒。他老人家背着孩子逃跑,累的痰喘咳嗽。这些个担惊受怕的日子,把他老人家累死。还有那年大饥荒……”
\\在河边,他们上了冰床。水生坐上去,抱着孩子,用大衣给她包好脚。女人站在床子后尾,撑起了竿。女人是撑冰床的好手,她逗着孩子说:“看你爹没出息,当了八年八路军,还得叫我撑冰床子送他!”\\她轻轻地跳上冰床子后尾,像一只雨后的蜻蜓爬上草叶。轻轻用竿子向后一点,冰床子前进了。大雾笼罩着水淀,只有眼前几丈远的冰道可以望见。河两岸残留的芦苇上的霜花飒飒飘落,衣服上立时变成银白色。她用一块长的黑布紧紧把头发包住,脸冻得通红,嘴里却冒着热气。她连撑几竿,然后直起身子来,向水生一笑。小小的冰床像离开了强弩的箭,摧起的冰屑,在它前面打起团团的旋花。前面有一条窄窄的水沟,水在冰缝里汹汹地流,她只说了一声“小心”,两脚轻轻地一用劲,冰床就像受了惊的小蛇一样,抬起头来,窜过去了。\\水生提醒她说:“你慢一些,疯了?”女人擦一擦脸上的冰雪和汗,笑着说: “同志!我送你到战场上去呀,你倒说慢一些!”\\“擦破了鼻子就不闹了。”“不会。这是从小玩熟了的东西,今天更不会。在这八年里面,你知道我用这床子,送过多少次八路军?”\\冰床在霜雾里飞行。“你把我送到丁家坞,”水生说,“到那里,我就可以找到队伍了。”\\女人呆望着丈夫。停了一会,才说:“你知道,我现在心里很乱。八年才见到你,你只在家呆了不到多半夜的工夫。我为什么撑的这么块?为什么着急把你送到战场上去?我是想,快快打走敌人,你才能快快地回家。”\\冰床滑进水淀中央,这里是没有边际的冰场。太阳从冰面上升起来,冲开了雾,形成一条红色的胡同,扑到这里来,照在冰床上。女人说:“爹活着的时候常说,日本人在这里,水生出去是打开一条活路,打开了这条路,我们就能活。你记着爹的话,不要为家里的事分心,好好打仗,我等你回来。”\\在杨柳树环绕的丁家坞村边,水生下了冰床。\\女人忍住泪,笑着说:“快去吧你!记着,好好打仗,快回来,我们等着你的胜利消息。”
\songti


\subsection{『形象』``赏析该描写在刻画形象上的妙处''}
\large 对于提问赏析描写在刻画形象上的妙处的题目,要注意以下几个点:\par
\df{拆分提干}\nm 分析哪个词对应了什么形象; \\
\df{立体地塑造了...}.
\subsubsection{例题}
\large 
孙犁的文字,"寄至味于淡薄".请以水生夫妻炕头对话为例对此加以赏析.\\
\texttt{水生以事业以家国大事为重,而在外因为吃鱼这一日常小事想起家中的日常生活,想起妻子,符合战士身份(1);女人想起水生,因为她生活艰辛,牵挂家中的顶梁柱在外打仗生死未卜,因此她无论白天夜晚都会想念他,情深义重,符合乡村女子的个性身份和生活(1);而女人听闻丈夫要走,虽然不舍但仍然送他,体现她深明大义(从另一个角度分析),因此人物形象丰满立体(1);语言平淡但体现出夫妻之间情深义重以及以国家为重的深厚情感(1)。}
\nm \fangsong 
\\\centerline{原文见上}
\\女人歪在炕上,笑着问:``说真的,这八九年,你想起过我吗?'' ``想过。''``怎么想法?''她逼着问
\\``临过平汉路的那天夜里,我宿在一家小店,小店里有个鱼贩子是咱们乡亲。我买了一包小鱼下饭,吃着那鱼,就想起了你。''
\\ ``胡说。还有吗?''``没有了。你知道我是出门打仗去了,不是专门想你去了。''
\\ ``我们可常常想你,黑夜白日。''她突然支着身子坐起来,问:``你能在家住几天?''
\\ ``就这一晚上。我是请假绕道来看你的。''``为什么不早些说?''``还没顾着啊!''
\\ 女人呆了。她低下头去,无力地仄在炕上。过了半天,她说:``那么就赶快休息吧,明天我撑着冰床子去送你。''〔乙〕
\songti \large

\subsection{『虚实』修辞之虚实}
虚实结合是一种常用的比喻手法.

\subsubsection{例题}
本段有何表达效果?
\nm\fangsong
\\寻梦?撑一支长篙,
\\向青草更青处漫溯;
\\满载一船星辉,
\\在星辉斑斓里放歌。
\large\songti
\\\texttt{本段虚实结合,描写长篙青草于满船星辉的场面,营造出梦幻般的氛围,表达了诗人心中自由自在的喜悦之情,将诗歌推向高潮.}


\newpage

\section{说理与议论}

\subsection{『举例』``请分析举例论证的效果''}
在这类题目中, 答题时必须包括\par 
\df{举了什么例子}, \par 
\df{例子阐述了什么观点}以及\par
\df{这样写有什么表达效果}\nm 典型事例,具有典型性.
\subsubsection{例题}
\large \songti
分析第一段以颜回为例说理的作用(3分).\\
\texttt{本段例举了颜回虽屈居于陋巷, 无施于事, 无见于言, 却不妨碍众人的推尊与后人的视之为圣人(1), 为前文只要修好身就能成为圣人的观点做了突出强调(1), 运用典型事例, 具有典型性(1).}
\nm \fangsong \par 草木鸟兽之为物,众人之为人,其为生虽异,而为死则同,一归于腐坏澌尽泯灭而已。而众人之中,有圣贤者,固亦生且死于其间,而独异于草木鸟兽众人者,虽死而不朽,逾远而弥存也。其所以为圣贤者,修之于身,施之于事,见之于言,是三者所以能不朽而存也。修于身者,无所不获;施于事者,有得有不得焉;其见于言者,则又有能有不能也。施于事矣,不见于言可也。自诗书史记所传,其人岂必皆能言之士哉?修于身矣,而不施于事,不见于言,亦可也。孔子弟子,有能政事者矣,有能言语者矣。若颜回者,在陋巷曲肱饥卧而已,其群居则默然终日如愚人。然自当时群弟子皆推尊之,以为不敢望而及。而后世更百千岁,亦未有能及之者。其不朽而存者,固不待施于事,况于言乎?\par 予读班固艺文志,唐四库书目,见其所列,自三代秦汉以来,著书之士,多者至百余篇,少者犹三、四十篇,其人不可胜数;而散亡磨灭,百不一、二存焉。予窃悲其人,文章丽矣,言语工矣,无异草木荣华之飘风,鸟兽好音之过耳也。方其用心与力之劳,亦何异众人之汲汲营营? 而忽然以死者,虽有迟有速,而卒与三者同归于泯灭,夫言之不可恃也盖如此。今之学者,莫不慕古圣贤之不朽,而勤一世以尽心于文字间者,皆可悲也!\par 东阳徐生,少从予学,为文章,稍稍见称于人。既去,而与群士试于礼部,得高第,由是知名。其文辞日进,如水涌而山出。予欲摧其盛气而勉其思也,故于其归,告以是言。然予固亦喜为文辞者,亦因以自警焉。
\songti


\subsection{『类比』``类比说理的特色是?''}
对于考察类比说理的题目需注意\par
\df{类比的本体和喻体}\par
\df{表达效果}\nm (抽象概念具象化)(生动形象)\par
\df{与文章主旨的关联}


\subsection{『说服』``哪个论证更有说服力?"}
\large 对于提问说服力的题目,要注意以下若干点:\par
\df{角度全面}\\
\df{语言严谨}\\
\df{论证手法}\\
\df{说理方法}\nm 例证,典型性\\
\df{因果}\nm 逻辑是否能自洽,核心概念内涵外延是否对应逻辑链\\
\df{修辞手法}\nm 例如:排比\\
\df{情景}\\
\df{角度}\\
如果原文有比较,答题过程中也应该注意比较.
\subsubsection{例题} 
\large \songti 第三段林劝后主夺回淮南诸州的话很有说服力,请分析其原因.\\ 
\texttt{先分析双方的形式:宋淮南各州兵力薄弱,疲乏,而我有思旧之民.指出此为恢复故境的可乘之机;其次具体详细地陈述了自己的计划:如何举兵,举兵之时向外界报告其为外叛;最后预设了起兵之后成与不成的不同应对,以解除后主的后顾之忧.}\\
\blk
\nm \fangsong 林仁肇,建阳仁,事闽味裨将,沉毅果敢,文身为虎,军中惟之林虎子,\\闽亡,久不见用。会州侵淮南,元宗拔为将,时周人正阳浮桥初城,扼援师道路,仁肇率敢死士千人,以舟实薪刍,乘风举火焚桥,周将张永德来争,会风回,火不得施,我兵少却,永德鼓噪乘之,遂败,仁肇独骑一马为殿,永德引弓射之,屡将中,仁肇辄格去,永德惊曰:「此壮士,不可逼也。」遂舍之而还,\\开宝中,密言于后主曰:「宋淮南诸州,戍守单弱,而连年出兵,灭蜀,平荆湖,今又取岭表,往返数千里,师旅罢弊,此在兵家为有可乘之势,请假臣兵数万,出寿春,渡淮,据正阳,因思旧之民以复故境,彼纵来援,吾形势已固,必不得志,兵起之日,请以臣举兵外叛闻,事成,国家衅其利,不成,族臣家,明陛下不预谋,后主惧不敢从,\\时皇甫继勋朱全贇掌兵柄,忌仁肇雄略,谋有以中之,会朝贡使自京师回,使使言仁肇密通中朝,见其画像于禁中,且已为筑大第,以待其至。后主方任继勋等惑其言使仁持鸩往毒之,俄卒。\\初,仁肇尤味陈乔所知,至时,乔叹曰:国势如此,而杀忠臣,吾不知所税驾也,然不能白其诬,仁肇卒,逾年,后主遂见讨,又逾年,国为墟矣。\\
\rightline{选自陆游<南唐书>列传第十一,有删改}
\songti


\newpage

\section{结构,思想,评价与概括}

\subsection{『构思』``请从构思角度赏析''}
对于提问构思的题目尤其要注意情感线索.例如,在文章<良宵>中,"鹅"这一物象贯穿全文,是一个十分重要线索.对于"鹅"的分析要注意其出现的段落与情感线索.我们需要考虑它的\textbf{来历},\textbf{经历},与它的\textbf{得失是如何推动情节发展的}.:\par
\df{结构}\par
\df{主旨}\par
\df{为什么能突出主旨}\\
\subsubsection{例题}
作品围绕``窥看自己的心魂''与自我对话,请从构思角度对此作赏析.\\
\texttt{本文首先提出心中难解的关于生死和写作的三个问题,并对自我思考过程展开反思.从尝试写作的试探,到沉迷写作的焦虑,最后到在对写作的坚持中获得与自己和解与释然,有深度.}
\subsubsection{例题}
小说中的"鹅"在全文构思中有重要作用,请加以赏析.\\
\texttt{鹅本是被人抛弃,由老太太捡来的,陪伴了她十三年,是她的精神寄托;鹅是老太太和孩子的矛盾冲突点(行文线索);鹅被孩子偷走,杀死,直接引发老太太和孩子的冲突,后老太太病倒,引出孩子给她做饭,推动了情节发展.为后文老太太消除误会照顾孩子做铺垫,突出老太太善良仁慈,抒发作者对善行义事的赞美.}
\subsubsection{例题}
本文重点写村民就医,而二三两段描绘集市场景,这种安排体现了作者的巧妙构思,请加以赏析.\\
\texttt{集市的场景让村民们去医院场景,多角度营造了充满活力和谐自然的乡村风貌,新颖独特.反映了到处充满生活气息的乡村.}
\subsubsection{例题}
文章标题为``风吹云动'',而文中较多笔墨写到雨,请从构思角度加以赏析.\\
\texttt{风吹云动,云除了云卷云舒飘动外,更呈现雨的动态身心状态生活.而雨是云所变,是云的另一种形态,也是具有生机的事物,使大地得以新生,也象征了人生的去旧迎新,引出后文中对人生意义的思考:云有淡泊有猛烈,而许多人的人生只有一种,难以拥有淡泊卷舒的境界,深化了主旨.}
\nm \fangsong 
\\\centerline{我与地坛}\\\centerline{史铁生}
\\设若有一位园神,他一定早已注意到了,这么多年我在这园里坐着,有时候是轻松快乐的,有时候是沉郁苦罔的,有时候优哉游哉,有时候栖惶落寞,有时候平静而且自信,有时候又软弱,又迷茫。其实总共只有三个问题交替着来骚扰我,来陪伴我。第一个是要不要去死?第二个是为什么活?第三个,我干嘛要写作? 现在让我看看,它们迄今都是怎样编织在一起的吧.
\\你说,你看穿了死是一件无需乎着急去做的事,是一件无论怎样耽搁也不会错过的事,便决定活下去试试?是的,至少这是很关健的因素。为什么要活下去试试呢?好像仅仅是因为不甘心,机会难得,不试白不试,腿反正是完了,一切仿佛都要完了,但死神很守信用,试一试不会额外再有什么损失。说不定倒有额外的好处呢是不是?我说过,这一来我轻松多了,自由多了。为什么要写作呢? 作家是两个被人看重的字,这谁都知道。为了让那个躲在园子深处坐轮椅的人,有朝一日在别人眼里也稍微有点光彩,在众人眼里也能有个位置,哪怕那时再去死呢也就多少说得过去了,开始的时候就是这样想,这不用保密,这些现在不用保密了。 
\\我带着本子和笔,到园中找一个最不为人打扰的角落,偷偷地写。那个爱唱歌的小伙子在不远的地方一直唱。要是有人走过来,我就把本子合上把笔叼在嘴里。我怕写不成反落得尴尬。我很要面子。可是你写成了,而且发表了。人家说我写的还不坏,他们甚至说:真没想到你写得这么好。我心说你们没想到的事还多着呢。我确实有整整一宿高兴得没合眼。我很想让那个唱歌的小伙子知道,因为他的歌也毕竟是唱得不错。我告诉我的长跑家朋友的时候,那个中年女工程师正优雅地在园中穿行;长跑家很激动,他说好吧,我玩命跑。你玩命写。
\\这一来你中了魔了,整天都在想哪一件事可以写,哪一个人可以让你写成小说。是中了魔了,我走到哪儿想到哪儿,在人山人海里只寻找小说,要是有一种小说试剂就好了,见人就滴两滴看他是不是一篇小说,要是有一种小说显影液就好了,把它泼满全世界看看都是哪儿有小说,中了魔了,那时我完全是为了写作活着。结果你又发表了几篇,并且出了一点小名,可这时你越来越感到恐慌。我忽然觉得自己活得像个人质,刚刚有点像个人了却又过了头,像个人质,被一个什么阴谋抓了来当人质,不走哪天被处决,不定哪天就完蛋。你担心要不了多久你就会文思枯竭,那样你就又完了。凭什么我总能写出小说来呢?凭什么那些适合作小说的生活素材就总能送到一个截瘫者跟前来呢?人家满世界跑都有枯竭的危险,而我坐在这园子里凭什么可以一篇接一篇地写呢?
\\你又想到死了。我想见好就收吧。当一名人质实在是太累了太紧张了,太朝不保夕了。我为写作而活下来,要是写作到底不是我应该干的事,我想我再活下去是不是太骨傻气了?你这么想着你却还在绞尽脑汁地想写。我好歹又拧出点水来,从一条快要晒干的毛巾上。恐慌日甚一日,随时可能完蛋的感觉比完蛋本身可怕多了,所谓不怕贼偷就怕贼惦记,我想人不如死了好,不如不出生的好,不如压根儿没有这个世界的好。可你并没有去死。我又想到那是一件不必着急的事。可是不必着急的事并不证明是一件必要拖延的事呀?你总是决定活下来,这说明什么?是的,我还是想活。
\\人为什么活着?因为人想活着,说到底是这么回事,人真正的名字叫作:欲望。可我不怕死,有时候我真的不怕死。有时候, ---说对了。不怕死和想去死是两回事,有时候不怕死的人是有的,一生下来就不怕死的人是没有的。我有时候倒是伯活。可是怕活不等于不想活呀?可我为什么还想活呢?因为你还想得到点什么、你觉得你还是可以得到点什么的,比如说爱情,比如说,价值之类,人真正的名字叫欲望。这不 对吗? 我不该得到点什么吗?没说不该。可我为什么活得恐慌,就像个人质?后来你明自了,你明白你错了,活着不是为了写作,而写作是为了活着。你明自了这一点是在一个挺滑稽的时刻。那天你又说你不如死了好,你的一个朋友劝你:你不能死,你还得写呢,还有好多好作品等着你去写呢。这时候你忽然明白了,你说:只是因为我活着,我才不得不写作。或者说只是因为你还想活下去,你才不得不写作。是的,这样说过之后我竟然不那么恐慌了。就像你看穿了死之后所得的那份轻松?一个人质报复一场阴谋的最有效的办法是把自己杀死。我看出我得先把我杀死在市场上,那样我就不用参加抢购题材的风潮了。你还写吗?还写。你真的不得不写吗?人都忍不住要为生存找一些牢靠的理由。你不担心你会枯竭了?我不知道,不过我想,活着的问题在死前是完不了的。
\\这下好了,您不再恐谎了不再是个人质了,您自由了。算了吧你,我怎么可能自由呢?别忘了人真正的名字是:欲望。所以您得知道,消灭恐慌的最有效的办法就是消灭欲望。可是我还知道,消灭人性的最有效的办法也是消灭欲望。那么,是消灭欲望同时也消灭恐慌 呢?还是保留欲望同时也保留人生?我在这园子里坐着,我听见园神告诉我,每一个有激情的演员都难免是一个人质。每一个懂得欣赏的观众都巧妙地粉碎了一场阴谋。每一个乏味的演员都是因为他老以为这戏剧与自己无关。 每一个倒霉的观众都是因为他总是坐得离舞合太近了。
\\我在这园子里坐着,园神成年累月地对我说:孩子,这不是别的,这是你的罪孽和福扯。
\blkx
\\\centerline{良宵}\\  \centerline{张楚}\\
她刚搬到麻湾时,村人并未觉得有何异样。这只是位干净的老太太,衣着朴素,脸上一水褶子,梳了低低的发髻,站在樱桃树下,束手束脚,竟有几分与年岁不相称的羞怯。隔壁妇人偶来瞅几眼,闲聊几句,才晓得是村里王静生的远房姨妈,想起要到乡下住上段时日,这才劳烦外甥在村西租了三间瓦房。行李也不甚多,几床被褥,一只泛黄的皮箱。随行只有一只白鹅。
\\好事的村妇们,借串门的名义在炕沿上东拉西扯。可这老太太,安静得像一只猫,也不插嘴。问她儿女几个?她说,两儿一女。问她老伴是否健在?她说,去世二十多年了。闲妇们渐渐没了兴致,不怎么来往。
\\那天从村西的土岗下过,见一孩子在前边跑,一帮孩子在身后追。那孩子蹽得比野兔子快,转眼就从她身边刮过,直刮到那岗上。那帮孩子呢,也就不再追,只在岗下骂个一通,才怏怏散去。老太太斜眼见那土岗上隐约探出个圆头,小心逡巡着岗下。见老太太望他,竟俯身捡起块土坷垃不偏不倚扔她额头上。老太太摸了摸额头,朝那岗上望去,孩子就不见了。
\\午后,老太太喝了碗稀饭,猫进被窝,看电视。过堂屋传来电饭锅被揭开的滋啦声,饭菜入嗓猛然吞咽的咕咚声……她蹑手蹑脚踱到庭院,见岗上那个孩子在往外翻墙。老太太默然看了片刻回了房。
\\翌日出门,买了冷鲜肉,切姜剥蒜,配了红椒、桂圆、八角、茴香,用高压锅将肉焖了。肉香四处散了开去,老太太眯眼打起盹儿来。等睁开眼,天已大黑,去过堂屋看炖的肉,明显是吃剩的。老太太竟有些隐隐的得意,方沉沉睡去。
\\次日早起,坐到屋檐下晒太阳,晒着晒着有些恶心,吞了几粒药片,倒头睡起来。醒来时太阳已爬上屋檐,却发现老鹅没了。
\\这老鹅,跟了她十三年,从小区门口捡的。小小一团鹅黄,谁承想竟长成偌大一只呢?儿女们是极少来的,通常只有她和它。想说话了和它唠叨两句,生气了就踹它两脚,它不记仇,依旧影子似的随着她,腻着她。
\\老太太在院子四周搜寻一番,仍没得踪迹。猛然想起那孩子,心就咯噔了一下。
\\那晚,她早早在过堂屋候了。果不其然孩子来了。当他在灶台上翻寻时,她一把就攥了他胳膊,问道:“是不是把鹅偷走了?”孩子点点头。她想也没想就在他后脑勺儿扇了一巴掌。“是不是把鹅给吃了?”孩子又是点点头。顺势拎了把刷锅的炊具,捋起他衣袖就抽打起来。抽着抽着便瞧得他胳膊上全是银元大小的红斑,一圈连一圈,看得心里麻麻幽幽,索性撒了他,一屁股坐在灶台上,默默盯了他半晌,摆摆手说:“你走吧,走吧。以后不要再来了。”孩子一愣,没有动,只嘟囔道:“我奶奶死了……我杀了它祭祀……”老太太不再搭理他,转身回了屋,和衣躺下。
\\一躺就是两天。再次睁开眼,屋里灯怎么就亮了。炕沿上摆着副碗筷,碗里尚冒着热气,是碗疙瘩汤。香油花浮着,白鸡蛋卧着。老太太心里热了下,吸溜起来。还好,病隔了一夜就痊愈了。
\\那天晚上,老太太喝完了汤,耳畔便传来谁家的收音机正在唱《春闺梦》,是张氏与丈夫王恢互诉衷肠那一场。听着听着,她不禁轻声唱将起来:
\\去时陌上花如锦,今日楼头柳又青!可怜侬在深闺等,海棠开日到如今。门环偶响疑投信,市语微哗虑变生。因何一去无音信?不管我家中肠断的人。
\\“咕咚”一声闷响,她才猛然梦醒,身子打个激灵,朝墙边看去,那孩子从墙头跌了下来。
\\“我……我……”男孩诺诺道,“我只是来瞧瞧,你的病好了没有。那天晚上,你的头比开水还热……”老太太领男孩进屋,给他热了排骨和米饭。
\\随后几日,男孩都过来共进晚餐。孩子通常只闷了头扒饭,很少动筷子搛菜。吃一阵偶然抬头,老太太便往他碗里搛一箸菜,孩子也搛了肉丁或腊肠,犹犹豫豫着往老太太碗里塞。老太太就笑。
\\当日晌午刚过,王静生就来了。王静生说,关于她跟孩子的事,他听别人说了。孩子爸妈、爷爷早死了,奶奶前几天也死了。孩子的病是父母遗传的艾滋病。那晚,老太太做好了饭菜,孩子却没来。
\\儿子第二天到了麻湾。老太太正在炕上收拾皮箱,儿子说:“哎,我真是白着急了,原来你已经准备回去了啊?这个礼拜日就是你寿日,香港的李老板做了你一辈子的戏迷,专门从香港飞来给你庆祝,光赞助费就掏二十万。饭店呢,就定在凯撒大酒店,省电视台要全程录像呢。”
\\老太太看他一眼,抽出皮箱拉杆,拍了拍儿子的肩,就朝土岗走去。儿子一见,蹙着眉喊:“妈!出租车在村东呢!”老太太大抵聋了,只顾弯着脊背拉着皮箱朝前走。儿子小跑着过去,在母亲身后边走边絮叨:“不瞒你说,赞助费说是二十万,其实给了五十万!不就听你唱两句《春闺梦》和《锁麟囊》?人家拿你当宝,傲气值几个钱呢?”
\\老太太径直走到了岗下,伸手擦了擦汗,将皮箱扔在土岗那厢,朝坡走去。这条坡不长,但是陡。老太太弯下腰身,晃晃悠悠往上爬,当眼前蓦然出现一只瘦骨嶙峋的小手时,她不禁抬起脖子瞅了瞅。当孩子的小手紧攥住她的掌心时,老太太身上忽就有了气力,手脚在瞬间就热了起来。有那么片刻,老太太确信双腿其实就踏在棉花般洁净干燥的云朵里,每向上微微跨一小步,就离天空和星辰近了半尺。
\\\rightline{(节选自《天涯》2012年第6期,有删改)}
\blkx
\nm \fangsong 
\\(文章见『比喻』一节) 
\blkx
\large \songti

\subsection{『排序』关于排序的题目}
\df{横线前后的文本}\\
\df{内容}\nm 话题\\
\df{逻辑}\nm 时间先后认知过程,逻辑与前文的呼应\\
\df{标点}\\
\df{试排}


\subsection{『判词』人物传记特色评价词}
我们经常能够在做人物传记题目时遇到诸如``坚正''之类的评价性词语, 进而要求分析人物. 这里要注意小词放大的技巧, 例如: \LARGE 坚 \large 持操守, \LARGE 正\large 直讲义. 注意要点\par
\df{小词放大}\par
\df{事例一概括与品格一}\par
\df{事例概括二与品格二}

\subsection{『事迹』``请概括人物事迹''}
\large
概括人物事迹时需要答道\par
\df{人物事迹一}\\
\df{人物事迹二}\\
\df{人物品格}\\
其中,特别需要注意一类伪装的题目
\df{侧面体现,烘托,说明}\\
如下:\\
\subsubsection{例题}
分析第五段的作用.\\
\texttt{最后一段先写陈桥在林仁肇被杀之后的悲叹,从侧面体现出林对朝廷的忠诚与被杀的宽屈;后写林死后不久南唐被灭的的事实,侧面体现出林对朝廷的重要.丰满了人物形象.对上文的补充叙述说明,是篇章结构的重要组成部分.蕴含了作者对林的同情.}\\
\nm \fangsong 
(文章见『说服』一节)


\subsection{『思情』``请分析全文的思想情感''}
分析时要特别注意题干中是说\df{具体分析}还是\df{大致概括}.
\subsubsectionmark{例题}
第16段最后两句``云是云,我们是我们.云不是云,我们不是我们.''意蕴丰富,请结合全文,对此加以评析.\\
\texttt{本句为作者的感慨,阐明了人生在世要学会从容淡定,淡雅淡泊的.对树立我们的生活态度很有启发.}
\nm \fangsong
\\(文章见『比喻』一节)
\large \songti

\subsection{『情变』``作者情感态度是怎么转变的?''}
对于情感题, 我们应注意\par
\df{全文逐段体会}\par
\df{找议论的句子}\nm (判断, 表推测)\par
\df{直接找情感关键词}

\subsubsection{例题}
\\小说第1段她仅仅是对他``点头致意'',第15段她却``直直地冲他微笑''.有人认为这一转变缺少铺垫,不切实际.对此你是否认同,说说你的看法. 
\\ \texttt{我不认同.奥一开始只是点头致意,表达了最基本的礼仪;随后因被窥视而感到不满;后来又因为里桑谨慎礼貌的行为留下了美好的印象,由此推测他是一位会忍耐,尊重的男人;最后她又因为他的骑士风度,从而想象不同情景下的他,产生海市蜃楼般的美好幻想.由此可见最后"直直地冲他微笑"有着充分的铺垫,是自然的心理变化的体现.(全文逐段体会)}
\subsubsection{例题}
\\第9段画线句运用动作细节和语言细节,刻画了老太太怎样的心理变化过程。请简要分析。
\\ \texttt{老太太首先"拾"炊具,"捋"衣袖,"抽"起来,后"心理麻幽幽的",并"撒",\\"默默"让他走,最后"不再搭理",体现了老太太由爱鹅被杀的愤怒到心疼男孩的不忍,再到心中矛盾纠结,最后无可奈何.}
\nm \fangsong 
\\\centerline{玻璃边界}
\\ \begin{center} 1 \end{center} \\当利桑德罗与奥德丽目光相遇时,她点头致意,就像出于礼貌问候一位餐厅服务生,比问候公寓楼的门卫还要少一分热情……利桑德罗已经擦净了第一扇玻璃窗,正是奥德丽办公室的那扇,随着他慢慢除去灰尘形成的薄膜,她逐渐显现,起初遥远而朦胧,随后便一点点靠近,由于玻璃越来越清澈,她分毫未动却越来越近。就像调整相机的焦距,就像慢慢将她据为己有。\\玻璃的透明渐渐揭开她的面纱。办公室的灯光从身后照亮她的头部,为她栗色的头发笼罩上一层麦田般的柔美和动感,麦穗与如饰带般落于颈后的美丽金黄的麻花辫纠缠。光线聚集于后颈,当她将浅色的柔软辫子拨到一边时,后颈上的光照亮了从背部蜿蜒向上的每一根金黄的绒毛,就像一把种子,即将在编织的发束里找到土壤。 \\她伏案工作着,对他无动于衷,对他人的工作无动于衷,那种卑躬屈膝的手工劳动,与她的截然不同。她正努力为百事可乐找一句精彩的、引人注目、朗朗上口的广告语。他感到不自在,担心自己手臂在玻璃上的挥舞使她分神。如果她抬起头,会是因为工人的打扰而一脸愤怒吗? \\如果她再次看他,会用什么样的眼神? \\``上帝啊,''她低声自言自语,``他们提醒过我会有工人来。但愿这个男人没有在观察我。我感觉在被窥视。我有点生气了,没法集中精力。'' \\ \begin{center} 2 \end{center} \\她抬起头,碰上了利桑德罗的目光。她想要发怒却没能做到。那张脸上有种东西令她吃了一惊。一开始,她没有注意他外表的细节。令她战栗的是别的东西。某种她几乎从未在男人身上见过的东西。她在自己的词汇表里拼命搜寻,作为一个以遣词造句为职业的人,她寻找着一个词汇,来形容这个办公室玻璃清洁工的态度和面孔。\\在一闪念间她找到了——礼貌。在这个男人的身上,在他的态度、距离感、点头的方式与奇妙地混杂着忧伤和欢乐的目光中的那种东西,是礼貌,难以置信地毫无粗俗的痕迹。 \\“这个男人,”她想,“他绝不会在凌晨两点钟歇斯底里地打电话请求原谅,他会忍耐。他会尊重我的孤独,我也会尊重他的。” \ “这个男人会为你做什么?”她马上自问。“他会请我吃晚饭,然后送我到家门口。他不会让我在夜里独自叫出租车离开。” \\正当她抬起目光、神不守舍之时,他在转瞬之间看见了她深邃的栗色大眼睛。他马上垂下目光,继续工作,但与此同时他想起她微笑了。这是他的想象吗?还是真的?他鼓起勇气望向她。女人对他微笑,非常短促,非常礼貌,然后就低下头继续工作。\\一个眼神足矣。他没想到会在一个美国女人的眼睛里看到忧郁。人们说她们都很坚强,很自信,很专业,很守时,不是说所有的墨西哥女人都软弱、摇摆、随性、拖沓,不,完全不是。问题在于,一个会在星期六来工作的女人可能是各种样子,也许温柔,也许亲热,但唯独不该是忧郁的。利桑德罗清楚地在这个女人的眼神里看到了忧郁。她怀着悲伤,也怀着渴望。她渴望着。这是她的眼神所诉说的:“我想要某种缺失的东西。” \\奥德丽不必要地把头压得很低,好躲进纸张文件中。这太荒唐了。她难道要爱上大街上第一个擦肩而过的男人,只为了和丈夫彻底分手,让他吸取教训,只是因为纯粹的反弹效应?那个工人很英俊,这是糟糕之处,他有着不寻常的几乎令人感到冒犯的骑士风度,不合时宜,仿佛在滥用他的弱势地位,但他同时有着明亮的眼睛,眼里流露出的悲伤和喜悦同样浓烈,他的皮肤呈橄榄色,鼻子短而尖,鼻翼翕动着,身形修长,卷发,年轻,胡须厚重。与他的丈夫迥然不同。他是——她又一次露出微笑——一个海市蜃楼。 \\ \begin{center} 3 \end{center} \\他也对她回以微笑。他的牙齿坚硬、洁白。利桑德罗想到,他极力避开了会使他在当他还是个有志青年时认识的人面前降低身份的工作。他曾接下一份在弗克拉尔餐馆做服务员的差事,当他不得不为一桌中学老同学服务时,场面十分难堪。所有人都事业有成,除了他。他令他们难堪,他们也令他难堪。他们不知道该怎么称呼他,对他说些什么。还记得和西蒙·玻利瓦尔队比赛的时候你进的那个球吗?这是他听到的最友善的话了,随之而来的是一阵令人尴尬的沉默。 \\他做不了办公室文员,从中学三年级起他就辍学了,不会速记法也不会用打字机。做出租车司机更不行。他嫉妒比他有钱的乘客,看不起比他穷的,墨西哥城混乱的交通令他发狂,让他火冒三丈,暴跳如雷,不停骂娘,变成各种自己不喜欢的样子……超市售货员,加油站雇员,他什么都做过,那是自然。不幸的是现在连这样的差事都没有了。他感恩能获得这份来美国的工作,感恩此刻正直视着他的这个女人的眼睛。 \\他并不知道,她不仅在看着他,也在想象他。她先他一步。她想象着各种情境下的他。她把铅笔放到牙齿间。他会喜欢什么体育运动?他看起来很强壮,很健美。电影,演员,他喜欢电影、歌剧、电视剧吗?他是那种会透露电影结局的人吗?当然不是。这一眼就看得出来。她直直地冲他微笑。她会忍不住给伴侣讲出电影、侦探小说的结尾,除了自己的故事,因为永远不知道会怎么结束。\\她头脑中的想法他也许已经猜到一二。他多想能坦率地告诉她,我不一样,不要相信外表,我不应该在做这些,这不是我,我不是你想象的那样。可他不能对玻璃说话,他只能爱上玻璃上的光,而光可以穿过玻璃,触碰她,光是他们共同所有。 \\(有删节)\\【注释】①选自短篇小说集《玻璃边界》。小说集通过九个短篇故事,生动形象地刻画了墨西与美国这对邻居在长达两百年的历史演变中形成的恩怨。作品出版于《北美自由贸易协定》正式生效后不久,此时美墨两国间的贸易壁垒有所弱化,然而两国人民间的交流依然存在一些难以解决的问题。②卡洛斯·富恩特斯(1928年11月11日-2012年5月15日):墨西哥作家,他的作品深刻刻画了墨西哥的历史和现实。由于对欧美文明的了解和对拉美落后现状的认识,比起其他的的拉美作家,富恩特斯作品中存在着更强烈的忧患意识。
\blkx \begin{center} 良宵(节选) \end{center}\\顺势拎了把刷锅的炊具,捋起他衣袖就抽打起来。抽着抽着便瞧得他胳膊上全是银元大小的红斑,一圈连一圈,看得心里麻麻幽幽,索性撒了他,一屁股坐在灶台上,默默盯了他半晌,摆摆手说:“你走吧,走吧。以后不要再来了。”孩子一愣,没有动,只嘟囔道:“我奶奶死了……我杀了它祭祀……”老太太不再搭理他,转身回了屋,和衣躺下。

\songti \nm 

\subsection{『意蕴』``请品读加点词并体会其中意蕴''}
对于这类题目,需要注意\\
\df{词语的本意}\\
\df{语境}\\
\df{这段话上下文的语境}
\subsubsection{例题}
\fangsong
然而阿Q虽然常优胜,却直待\CJKunderdot{蒙}赵太爷打他嘴巴之后,这才出了名。\\
\texttt{"蒙"意为承蒙. 挨打也像荣幸蒙恩,形象地刻画出阿Q与看客以丧失人格为代价而换来趋炎附势的变态心理,体现了长期以来的奴性人格.}
\nm \songti



\newpage
\part{临时笔记}

\subsection{了解命题}
需要注意的命题依据:\textbf{教材}.
教材中的学习提示尤其需要重视.在学习过程当中要非常有意识的去掌握每篇课文涉及到的文字、文学、文化的知识.在翻教材的过程当中查漏补短.
\\
\subsection{单元任务(粗体为高考考察过)}
\\
\begin{tabular}{|c|l|}
	\hline
	教材册次 & 读写任务 \\
	\hline
	必修上 & 札记,小说点评,\textbf{推荐书},文学短评,\textbf{访谈提纲},建议书,内容摘要,\textbf{调查报告},散文评点 \\
	\hline
	必修下 & 阐述事理,\textbf{招聘启事},宣传语,演讲稿,读书札记,\textbf{短评},\textbf{发言提纲},短论 \\
	\hline
	选必上 & 探究诸子言说风格, \textbf{鉴赏小说艺术},\textbf{评析观点},\textbf{概念},\textbf{论证} \\
	\hline
	选必中 & 读后感,辩论,报告文学特点,鉴赏札记,\textbf{人物短评},改写诗歌 \\
	\hline
	选必下 & \textbf{评价观点},\textbf{语言鉴赏札记},古代散文评点,论文摘要, 读书报告 \\
	\hline
\end{tabular}
\\
\subsection{征稿启事}
注意题干情景限制(表达目的,对象场合文体)与用词用语限制(修辞对象,谦辞敬语).

\subsection{评点}
所有题目都是答艺术手法的作用对象.
\subsubsection{例题}
从人物描写的角度,为<药>的第12段写一则评点.\\
\df{修辞对象是什么?}\\
\df{描写中体现了人物什么特征?用什么词语概括特征?}\\
\df{评点指向什么?(主旨)}



\section*{发言文本}
02:47
刚才陆老师给我交代了,今天我们上课的时间是9:30之前要结束。那么内容是比较多的,在之前和付老师交流,我们讲什么内容,因为现在高三学校处在二模前的复习阶段,那么今天我们回顾了一下去年和春考一模的一些复习要点,包括我们高三一模考的共性问题,所以我就想了一下,今天我们讲的主要的点落在备考综合应用题,为什么这样说?

因为高考是高厉害的选拔性考试,所谓高利害就是非常注重区分度,那么在高考的试卷当中区分度最好的当然是作文,那么你们老师给我的任务是讲阅读,阅读部分如果从大题来说,最好的区分度是史传的就是第五、第四大题,现在的当然每道阅读题比较好的区分度应该是综合运用题,而这个题目是非常考验同学们对文本的理解能力,语文知识的掌握能力和完成命题人给定的任务的能力,它是一个综合能力的考量。

所以在二模前的备考当中,我们就今天聚焦这个点,那么这个点是依据在哪?在你们的讲义上,其实我是做了一个梳理,我们今天讲的内容,我想我和你们老师做一个分工,因为老师讲的大多是文本的阅读以及依据题目来回答问题。

那么因为我的工作性质,我的任务是每年都要立像高一高二高三的监控考,我们上一周考了高一高二的监控考,我的任务更多的是命题,当然命题的后续是老师的阅卷,老师是怎么样采分的。

所以今天我们聚焦在怎么命题和高考后或者讲阅卷后怎么批卷子,怎么评价,把重点放在这两个点上,也就是说我们先作为考试,我们先支笔,了解怎么命题怎么评价,然后反观自己来自己来补漏来补足短板。

那么这一章课件主要是了解一下大致命题的一个宏观的策略,这个我们不做重点讲,在每一年的冲考前,考试院都会发一发一本叫做高考的考试手册,但是从今年你们这一届没有了以后也不会再有。

原来我们的命题我们总注意,主要是聚焦考试手册以为依据来命题,但是今年没有以后,他们的命题主要是依据三点,一是课标,课标我想在座同学们没有必要了解,老师会通过教学传递给各位。

第二就是教材,这是同学们要清楚的,特别是高三同学,高三同学已经读完了5本书,所以我们必须对高三高中5本教材单元后面的任务做一个梳理。

第三是21年的8月份,教育部出台一个高考命题指南,那么命题指南成为目前来说高考6套试卷的命题的1个什么?方向,这是一。

第二就是现在的考察考语文素养上面的4个维度,语言思维、审美文化。那么在这几年的命题当中,教育部有要求命题要求叫4性,综合性。首先是基础性,就是像默写、理解能力、基础性,还有4点就是像综合性、应用性、开放性,开放性是什么意思?从评价来说这个题目可以同意可以不同,就是可以两可回答叫开放性。

那么综合和应用其实它是放可以整合在一起的,那么其实今天我们给到同学们的是综合应用题,就说给个文本,不是这个文本当中所有题目都给你们做,我们就聚焦1~2个题目来,那么我们重点也不把文本解读作为一个重点来讲,那么由命题我们反思自己怎么样的应该查漏补短。

那么查漏补短首先是5本教材的学科知识,这个知识举个简单例子来说,比如说我们高一学到前赤壁赋,在座的高一同学都有印象,上个学期刚刚学完的,那么这篇文章如果我们读完以后,知识要掌握什么?首先是古代的赋体的主客问答的构思的特点,它是一种传统主客问答这种对话构思,这是第一。

第二就是赋体的语言特点,浅散结合呼成含义。那么就是说当我学完这篇课文以后,相关的知识卡片我全部都要掌握,我只举个例子,由此如果说有相关的问题,比如说我们给到同学们的大好时光,他用书信的方式就是一个主客问答,或者说是你我问答,其实是做的一个创造,并不是说有这么一个业务大姐有这么一个我,而是通过这样的一种问答,他来表达他想表达的情怀和他的道理,他的发现是这样的。

那么也就是当我们考察构思方式方法的时候,大家不要以为我没有学过,其实是我们运用高中学过的知识来解决问题。

那么你就来想一想在高中教材当中有哪些文体知识有没有掌握的,比如说21年21年有一个5分的诗歌鉴赏题,要求鉴赏虚实相生的手法,那么虚实相生手法其实是出现在选b以下第一单元当中的后面的学习提示,要求掌握虚实相生手法,也是这个单元的单元任务,那么诗歌的虚实相生,那么同学们马上反思一下,我有没有掌握他高考是要你他默认你前提是你掌握了,然后我要你运用运用来鉴赏这首诗歌。

好,这是我举两个例子告诉同学们,不管是高三的还是高一高二正在学习的同学,我们在学习过程当中要非常有意识的去掌握每篇课文涉及到的文字、文学、文化的知识,三至少是三个维度。好,那么在这一点当中,比如说我们说文体的读法,比如说今天我们要讲的答题的要点,以及综合能力探究题,这类题目我们应该怎么答?这是我们对照前面的命题要求来反思自己,结合我们的一模和春考。

因为春考成绩同学们都有了,来看一下我有哪些短板。那么在春考的阅卷当中,有阅卷老师也跟我们说出好卷子是非常难的,其实你们看一下你们的教材要求,其实教材要求是高于中考的卷子的,那么也就是说秋考卷不可能它的难易度会低于峰考,它是低不下去的,因为教材就是我们的命题依据。

那么也就是说我们在学习过程当中,我们是要非常清楚教材对我们有哪些要求,而命题人比如说命题组他6个老师命题组一半老师是大学老师,一半老师他是没有上过他不上课,他依据什么命题的翻教材,他不停的翻教材一翻,比如说我们的19小题,要求写文学短评,写人物短评对吧?

其实是说武传屈原列传那篇课文,单元那篇课文的单元任务,要求站在班固的视角写一则人物短评,那么你们有没有写过?

很多同学一看这个题目傻眼了,当时我们在讨论这个题目的时候,我们就说可能学生没有练过,命题人说的这是教材要求,不是我们杜撰,我们命题是有依据的。

我讲这个意思就是说同学们在特别是高三同学在备考时候,你翻一翻教材,到底教材有哪些东西我从来就没有变过,我就忘掉了或者是忘掉了,或者说我还没有掌握好,那么我们的要求上面4个字,查漏,补短是我们当下要做的事情,为了更好的应对我们的二模考试,因为二模也是一个诊断性考试,以此来检测我们知识的掌握,以及我们应对各种题型的能力。

那么这里我做了一个非常简要的梳理,把高中读本教材当中的后面的读写任务,这里我们不对教材做研究,说同学们手上都有这张表格,那么你们在看的时候,我这里的不同在我标红了一些任务对吧?

标红的就是高考考过的,那么还有一个并不意味着标红的不重复考,比如说举个很简单的例子,鉴赏小说艺术,这是选b上单元任务,比如说平息观点,这也是个任务,那么这种单元任务它其实经常反复重复出现在我们的考题当中,那么再比如说今年我们春考的人物短评,就是选自于刚才我提到的选b中,同学们高二同学正在学的要求,因为你们可能还没有学到权力上的书是吧?

还没有学这篇课文,课文后面就有人物短信。

那么由此比如说像其他的其他的读写任务,包括后面的这个像论文摘要,论文摘要我们这次练习当中有它也可能出现在我们今后的考试当中,那么我想这个是我们先对教材做一个梳理,其实这都是我们学过的。我们也默认各位特别是高三同学掌握的差不多了,那么你们只是一个运用的问题了。

好,那么接下去我们就来结合同学们的预习作业,来讲一下首先是命题的思路以及答题的评分要点。这个题目同学们是做过,特别是高三同学,那么他考察的其实是高一上第四单元的就是招聘请示当中的什么应用型的写作。

那么这个启示当中我们在读我命题强调两点,同学们以后读题特别关注命题题干的限制,限制第一就是某校文学社它的受众是同学学校的同学,直接的受众是文学社的这些成员们。

第二征稿启示它是一个书信体。

再有一点你在运用词语的时候,在这个语境当中,它的近千字的运用,褒贬义的异味,都是要我们在比如说他划线的这4个地方,他要你找出他表达恰当的一项,那么它不恰当在哪里?在这个语境当中,它的限制就是它的修辞对象,是不是能够修饰对象。第二就是他的褒贬。它的进迁其实在我们的必修上第四单元高高一同学刚刚学过的,应该是有相关的教材当中有相关的介绍,那么我想这个可能会出现在我们的选择题的应用题当中。

我们在看一下这篇文本这篇我们的文选的是选修教材,我不知道你们有没有高三同学有没有看过选修教材,我们的课程包括必修选选择性必修和选修,但是高三的同学高考的能力也就考到必修和选b这个考到我们课标当中的第四个层次,选修不要求考,但是我们可以把它作为比如说我命之人,我可以把这篇经典的小说作为一个给同学们掌握小说读法的样本来命题,那么这些小说它还有一个资料,就是他小说当中附了一个复旦大学吴中杰教授的平点,就是在右边有个平点。那么这篇课文其实出现在我们原来的沪教版的高一下的教材当中,用这个来考什么?如果说我们刚才看看刚才的这个表,大家可以看刚才那个表,这个表当中其实是同学们可以看一下评点这个要求,出现在选b上散文的品种有印象吗?还出现在选b上的评析观点,还出现在选b下高三的上学期的古代嗓子的评点,就是评点读写要求出现在贯穿在我们高中三年的教材当中,也就是说我们其实默认各位默认各位是练习过频点的,那么因为练习过评点,所以我们应该会读评点。

那么这一篇就是我们选了我们刚才提到的既要读文本,又要读吴中杰的评点,那么我们的问题写一则人物评点。

好,首先我问几个问题,就是评点的对象是什么?题干中限制了是人物描写的角度,第一个我们答题修辞对象是是谁?在这一段当中大家看一下华老栓和黑的人,我们先确定修辞对象,这是我们答题的第一个要点,所有题目都是答意识守法的作用对象,艺术手法的作用对象,这是我们首先要确定的依据。

第二就是在语言和动作描写当中,他写出了这个人物的什么特征?你用一两个词来概括它的特征,比如说华老三的老实胆小黑的人的贪婪凶残,那么你是不是习惯用这样的词语来描述人物?

这里我们说一下题外话,在史撰文当中,我们选的人物传的都是扬尘名利。大家读一下凉城,比如说屈原,比如说苏武就凉城名利,那么我们在当下的高三就应该梳理所有良辰明丽的形容词,2014年要用4字短语来概述人物的特征,那么这个时候人们就挖空心思短语到哪去找,如果记忆库当中我就有这些短语,储备考试时候我就信手拈来,哪一个能够对应得上标准。诚信。

那么现在我们就要比如这里我们是用贬义词、老实、胆小、贪婪、凶残来概述,对吧?然后评点还有指向什么?评点指向在这个人物描写当中画老栓和黑的人的交易。它其实是一个场景描写,是一个场景描写,那么这个场景描写作为评点者,站在读者立场,站在我阅读者立场,我来讲我的感受,所以我们在答题当中的3分,其实3个3分,我们还可以把它细化成加,比如说有分还能变成4分,分值越高,区分度越小。

我们讲这一点就是答题要点当中的三个,一,就是评分要点一,你是否确定了修辞对象,就是你是否聚焦了题干的限制,人物描写。第二点,你是否能够用简要的语言来概述描写出的人物特征?第三点,你是否能够站在读者立场来平息这个场景描写给你的阅读感受?比如说我们最后一句话,交易革命者鲜血的场景,让人悲哀,并避免引发人的思考。也就是说夏禹的死,他的血居然作为一个交易的货物,卖给了发老三给儿子治病。

我们今天来想一下人的血怎么能够治病,说明花老师的一种愚昧,他居然把家里所有的家产拿出来,你们可以看他拿出钱,摸了半天摸出一点钱对吧?然后拿去来交易,那么最后结果我们看到的小说总共有4个场景,它的构思有双线交织,场景转换、构思特点,最后的结果化老栓还是把儿子送进了坟场。

那么我们可以想见一下,这就是说当你看完以后,我们的评分要点,大家注意我标标的分值。

那么我现在我们还原一个场景,比如说像今年高考,如果没有疫情,正常是6月份高考,对吧?那么6月七六七号考试有完了以后,我们10号那天在华师大的机房里面来采样,采样是干什么?第一个就是把三分切分成0分、1分、2分3分,它是个主观题,但是主观题要客观化操作,为什么?为了一保证公平,能区分出语文成绩的语文能力的好坏,第二保证速度。

比如说我们去年的第七小题6分题总共是26位老师,你想象一下体育中的第一排第二排坐了26位同学,怎么能够保证这一题所有人的评分要点都能够公平的?

你凭什么说我写了5行字你给我3分,我写了3行字你给我5分,凭什么就凭我们产量,我的依据在于把评分要点细化,细化的依据就是刚才我提到的,好,这个我们这一题同学们可以看一下,后面的评分要点,我们是怎么样把它细化的。

那么也就是说其实我们在答题当中,我不知道你们的心态,比如说我来做监控考试的答题纸,那么这一题我在想是给3行还是4行还是给5行,如果给商行有一半同学写下,他就会写的密密麻麻,就打个箭头老师在上面,一般这个是作废的,没有耐心的,老师们不会看你上,因为扫描不到的方面,那么怎么样应该尽量的语言要简明,或者用一个成语叫言简意赅,语言简洁,意识完备。

那么就是说你们可以看一下我们的三个要点,很多同学特别会堆垃圾,同学你给他3行,你给他4行,他写4行,你给他5行写5行,给他66行写的满满的,写完以后阅卷老师来视察几天,就把这个语言来语言找当成要点。其实你答了5行,你还不如答3行的同学非常简洁明了,也就是说这个告诉我们在考试时候时间就是分数,时间来自于哪里?来自于你对于答题思路得清晰化。

好,我们继续。那么这个题目是一个情境题,情境题其实回应了我们高中教材当中的写发言稿读写任务。当然这个是写一个发言的提纲,因为要阐释选编的理由。那么现在我们来读题习惯当中有一个叫阅读经典,积累学法,做完题的同学有没有关注到这8个字?

如果说我来选鲁迅的药,那么我应该是独一家之一家,当然教材当中不仅仅鲁迅的小说文本就有一篇我们是借助于经典的小说,我更加能够把握鲁迅小说的写作特点和他的风格特点,比如说他的语言的精准幽默智慧,这是他的鲁迅的小说语言。

那么如果说我选把评点加进去,那么它更注重一个评点的法律关系,那么就是说我的答题要点,我要更加聚焦于读法和学法的回应,比如说我们举一个小慧的答题事例,小徐小慧,小慧的答题事例,他说先是观点的鲜明性,我建议选5要加上吴忠杰的评点,这是我的观点的鲜明性,因为是开放题,这是一个开放题。

那么下面就来说选药对于把握经验的作用。第一位这个药是鲁迅的代表作,以药为线索告知隐蔽主体,这是对于小鲁迅这篇小说的构思特点的把握。比如说我们从红楼梦它也是双线交织,对吧?那么再就是我们把评点加进去,那么我们读了吴中杰的评点,当然我是节选,因为这个小说原来它有4个场景,那么我们在命题当中整张高考卷的阅读量是6500字以下,所以我们做了删减,那么删减又要保证小说的完整性,所以什么?

选了非常经典的几个片段或几个场景,那么在药品点这一点上,它的评点包含了环境,包含了人物,还包含了一些自己读书的感受。就是说它的评点是涵盖了小说的所有的方方面面,人物、环境、场景,包括取名。你看鲁迅的小说,他取名很奇怪,化老栓,夏雨这种华夏合在一起,那么其他的人物他没有名字黑的人,是吧?当然还有康大师都总是用人的符号来代替他的,用借贷手法把它符号化。

那么下面一点就是关于要评点给你阅读小说,给你写作评点的提示,那么再有我们不要给我们积累经典,阅读经典、积累读法的作用,还有什么?一方面有了读了小药,我们能够从构思上,从主旨上我们来了解小这个鲁迅的作品风格,我们加上评点,我们帮助我们进一步来把握情节和主旨。

并且还有一个非常重要的就是能够帮助我们点拨我们如何评点小说,来注意我们建构阅读小说写作评审的经验和方法。

在后面一点上他选的是建议加上要评点,那么前面的小徐的答案是建议就选药,那么选药有选药的理由,这一点我们就做一个说明,开放题应该是不贬斥任何一种答案或者任何一种观点,而我们更重在评价了你理由和观点是否极强,这是一个非常重要的评价依据。

那么还有一点,这里我们也做一个补充,比如说5分体采样,刚才我们提到了评价的把它做一个细化对吧?那么还有一件事情就是采样还要做一件事情就是补充,比如说今天我们每天进第一天采样这一题,我们要采样500份卷子,比如说今年我们有6万考生采样500份卷子,那么采样的时候,我们给出的是答案示例,你们同学们可以看一下。

那么还有很多同学他对于文学解读有他独到的见解,所以我们采样还有一个任务就是补充同学们理解准确,答题符合题干限制的答案。

大家清楚了,也就是说在主观题的答题上,其实我们答案不完全是他不是唯一的,他一定是认同你们对于文本的准确理解,对于答题题干限制的要求,如果是有这些我们作为补充答案,所以每一年在第二天批卷有半天是采,半天是试批,是现在让比如说这一题26个人批,让26个老师统一思想,统一评分标准,这样的话尽量的包容认可同学们的合理答案。

那么这里我们同学们就可以做一个就是说我未必是按照你的答案是,所以我们在每次做试卷分析时候,我们都会给出很多的满分的同学们的答案,来让其他同学来学习笑话。

好,这个是关于药这篇小说,那么我们再看一下,一这一天的其实是我们高二监控考的事情,做成之前我们先做一个说明,就是红楼梦的阅读其实不是考证文本,那么为什么这么考?

我说一下我命题人的想法,这个学校记录下的8个单元,不管老师还是同学都会有1个困惑,就是说我们这个学期这么短时间8个单元来吗?

那么我记得我们在2017年的1月,就元旦的时候,我们到山东济南参加人教部人教社教育部的培训,但是我们很多老师都提出了这个困惑,那么当时的副主编尤伟老师是这样说,那么红楼梦不是读一年,是读高中三年,他说这三年你可以常读常新。

还有一点他的个人想法我比较认同,他说红楼梦不是为了高考三次分析,而是让你了解中国章鱼小说的一种读法,让你能够知道懂得中国文化,它是这样一个目的,当然今天我没有时间分享,其实我们在去年冲考阅卷当中浓缩就是精华吗?有一位同学专门以红楼梦作为一个样本写浓缩就是精华,写得非常精彩,也就是说其实红楼梦这本小说是你很好的写作素材,我不知道在座同学们是不是会常用它作为素材来写。

那么回到文本这个文本,还有第二个给同学们的读朝晖小说的一种读法,也就是我们读红楼梦还可以借助红楼文献帮助你理解小说的人物情节和主旨。所以在这篇文本当中,我们选了23回以及相关的文献作为一个互文,回应我们教材2/3的课文都是一课多篇这样一个编写理念,那么现在我们也发现在抽考古代散文二就是选了两篇文本,去年初考现代文,一也领了两篇多文阅读,也就是说多文其实也是以后会不断出现在我们试卷当中的一种选文特点。

那么在这个题目当中我们写了23回,我在说一下为什么选23回?我们先来看一下这张图片,顺便光顾一下红楼梦。高一的同学也顺便一,我不知道你们看到哪一回了。那么根据周永康先生的同学家他的研究,他认为保贷的感情发展可以分为这么几个阶段,也就是说在这部小说当中,宝宝代开的雄县作为一个主线,我们读小说是必须要掌握的,红楼将近96万字,你不可能所有的文字情节人物已经记住600。
说话人1 35:38
多个人物。
说话人3 35:39
我们要把握的是主要人物,主要情节,那么宝黛爱情它既是主要人物,又是主要情节。

在这个表格当中,同学们可以看一下23回在情感发展的第三个阶段,也就是宝玉和那些少女们刚刚搬进大观园的时候,二是怎么样?元妃省亲回家,回宫以后,他们在22回的末尾他们搬进了大观园,那么搬进大观园以后就是好在运营西乡这样一个情节,在这个情节当中,保贷对于自己在对方当中的位置,他们是一个试探期。

那么我们读这个小说,我把这个拿出来给你们看两点,第一点就是我们复习,高三同学复习,高二同学重温,高一同学正在读,应该对于他们的情感发展贡献有一个宏观的把握。

我考到哪一回我马上就要知道,我们应该在小说的前因后果当中分析人物,它不是雨天夜里的吗?不是,这个题目的区分度非常好,我们在高二监控法发现很好,为什么能够在整本书的情境当中答题的同学和只看影片的同学不一样,所以有同学就问,看长篇小说要不要记人物记情节?

我认为一定要记,你只有记了,因为它是反映中国乡土社会怎么样插税格局这样的一个特点。四大家族荣林二府它就属于插税格局,建立一个什么人物的网。那么再有一个我们下面还有一个题目是关联到小说的,大家看一下32分钟,大家看32分这里我没有标准,先标红一下,大家有没有印象?

因为等会有一个题目跟有关,第三十二回,当然第三十二回等会要我们要讲的是33回,有一个题目是33回,就是一笔多用的,那么我们先重温一下,我认为这个不是浪费时间,我们其实有必要重温一下,不管你读的怎么样,在32回当中,他们两人共处,终极我们不展开,那么我们有一个是考33回,33回就是去年高考的选择题,第二小题1/3保育挨打,因为这个是贾府盛衰事件的重要情节,所以他拿来跑。

那么三宝玉挨打以后,大家回顾一下有哪些人去看望宝玉,我们也不展开,当然包括待遇,但是戴玉哭红眼睛以后回到了房间,到了晚上宝义左思右想不放心,就派晴雯去送了,就怕。那么这一送就怕其实叫做送怕定情,这是一个幸福。

那么大家再回顾一下,看到第几期火也看一下,97回当二宝宝一宝钗在成婚的时候,黛玉在烧,就怕因为就怕上面有他写的绝句,这个绝技就是那天宝玉送他就他所写的,那么宝玉娶了宝钗,黛玉只有一死了之,烧就怕来叫断情一份。这个也是小说的什么?叫福脉千里。小说的一个情节构成不败谦虚,对吧?我不知道大家可现在我看同学们的反应,有些人在看着点点,说明什么呢?你对这个情节你是唤醒了你的记忆,有些人是很茫然说他对这个一点不了解,当然你可以做到这个高考这两三分我不要了,你可以这样说,我有一个同事他的孩子去查了一下中考的位置,中考的报考卷查你还去。那么他为什么你会发现只要差两分,你的排序在整个试点的排序往后退,到你明年比如说你考华政差一分的时候,你就会发现这三分我非常有必要,这个时候你才感觉到你你心里才有一种疼痛的感觉。

你考评说的无所谓,我不要这三个,但是我刚刚说了高考高利害选拔性一分两分就有几百人,所以这一我们重温唤醒你的记忆,让你对于这部小说宝黛的情感有一个重温的记忆,回到第五小题。

这一题其实你们可以看一下你那个表格是考什么?第一个好小说的艺术的建设构思艺术,第二考整本书,你马上就要把宝玉还原到情境当中,什么情境?刚刚搬进了大花园,他和宝他和黛玉之间是一个朦胧式看齐,谁也不知道自己在对方心中的位置到底分量轻重如何,那么谢灵运有这么一句话,这句话当中谢林业就是说是四者难并,什么叫四者难并?

这4件要同时兼顾是非常难的,同学们有体验吗?又是良辰美景。

当然如果今天天气在3月的天气,如果说是周末的时候春暖花开,那么你的跟一家人去那里耍,春天的什么花花草草当然是四者兼有,像天气呢世博文化公园,周末的时候我们也走了一圈,发现文化公园的草坪上全部都是帐篷,那么很多小朋友包括老年人,当然还有各种年龄的人在那里赏花,在那里喝咖啡,在那里还有遛狗的,他那里有个狗哥乐园,那么这就是四者兼有,现在我们回到这个情景当中,宝玉是不是这个时候有,那么你就要唤醒你的记忆。

大家想一想,天气刚好桃花盛开的景,是不是良辰美景,春花阳春三月,第二媒体当中还有谁?恰好黛玉在这里,而且黛玉属于个人在干嘛?黛玉葬花黛玉认为大观园是干净的,大观园的水也是干净的,只要流出了庆发达,流出了大观园水就是脏的,所以他不要把大观园的花流到外面去,要把它葬在大观园里面,大家有没有想清楚这个意思?
说话人1 44:08
这个意见。
说话人3 44:10
那么这个时候恰恰褒义来了,褒义在电影看会真记,也就是当事人晋书西厢记,那么宝玉向黛玉推荐了这本书,大家想见一下你们有没有看思道?宝钗券宝玉待遇不能女孩子不能看这件事有印象吗?那么这个时候待遇来看这本书有没有风险是有风险的,他能不能看?

第一个他能不能看,第二他喜不喜欢?好,想这两个问题,那么偏偏他们两个叫心灵的知音,真是心有灵犀,看来以后他一目十行,并且还记住了很多句子,就像宝玉说的连饭都不想吃,看你们现在看什么书连饭都不想吃,我想打游戏的伙伴不讲实话,完全沉浸在里面的情境当中了。

好,那么现在我们的问题是我们用谢斌这句话来评价23回在桃花树下共读西厢的锚定是否合适,评价的依据就是现役的这句话。

那么我们要一一对照,我们现在先来看一下,这是一个开放题,是否?那么这个答案我们给到的是不是?为什么说不适合?他说逝者难,自己说难,但这个时候的宝玉那是非常享受的一个情境,那么下面首先我们对于良辰美景的一分,大家看一下百分你要回应其他的限制,良辰美景,良辰阳春三月,美景桃花,落英缤纷,桃花缤纷的落下去。

还有两后面的两个赏心悦事。如果我们说这个时候黛玉他不喜欢保密的推荐,或者说骂这个保密,你怎么给我推荐这种情况,你看舅舅会骂他以假善来要挟他,或者一看一丢不看这有什么好看的,天天都是这样。

所以后面4个字恰恰能够印证此境的宝玉。

我们后对后面做一个分析,那么我们如果往下看你们的文本,我们给出的文本就是保利借此引用了张生的话,非常含蓄的巧妙的智慧的向黛玉表白,黛玉有了回忆,我们就对后面的4两个词来做阐释,最后是可以说是四者兼有。

当然有同学答案的时候,我不知道你们答案有没有说合适的,那么合适我也可以给你5分问题是你怎么样阐释?不是良辰美景,不是赏心悦事,这是你的理由,我们说理由要自洽,也就是说有同学把落英缤纷把它理解为长花落半,当然是可以的,对吧?花的败落你可以讲是非常美的,你看到落英飞在地下的,你有没有觉得青春流逝了,你也会这样想。

所以我们就是说在这个情景当中,宝玉心里是很多心小心思的,他要向黛玉表白他的心事,但是他不知道该怎么说,所以你说这不是赏心,也可以,问题是你如何来阐释你的理论,支撑你的观点?

好,我们接下来去看一下,下面这个就是我们说读红楼其实要借助大量的文献来阅读,当然可能同学们没有时间来阅读,如果说能引发非常少的时间来读人物的话,其实曹立波有一本就是红楼12金12金钗,是对你了解红楼首尔金山的保密非常有帮助,他写得非常的详尽,陶立波的有中文大学的一个教授,或者和王国荣的同谋荣誉性,我觉得这两个是可能性非常强的,对你了解能不能非常有帮助,当你没有时间,你就把原来的梳理梳理一下就行了,巩固一下,那么这一题是我们借助红楼文献当中的一个片段,我们要赏析的是在章回小说一般是一回两世,大家有印象吗?

一回两次,那么宝黛共读西厢以后,袭人来叫宝玉,说你出去教育宝玉就走了,走了以后黛玉回他的潇湘馆,那么在回家路上为萧山管路上,他听到12个戏子银关上当关上也在唱淡季,牡丹亭的唱词引发了黛玉的共鸣,那么这一段我们要赏析的是注意题干中有提示,保代共读,共读是赏心乐事,后面的待遇是读听牡丹节,如果说你看待遇的,读听那一段的,他的神态动作描写经常是落泪的,那么为什么会落泪?

你又要瞻前想到黛玉是继人名下,不像我们在座的,如果你住在外婆家,你现在没有感觉对吧?如果在乡土社会的外婆,注意你和外表奶奶可没对吧?是外婆家,所以他是寄人礼下,他是一种绅士之碑。

那么我们再看一下,如果我们来分析,我们要把共读到转为这两个情节的承转做一个概述,因为我们要先赏析构思图,要构思图要的修饰对象是什么呢?是由供到读承转。那么承转构思基调妙在哪几点?同学马上想一下,如果说我讲有3.3分,你会哪打哪三点?

构思之妙,妙在哪三点?就是你答题的方向也很清晰,哪三点呢?一般构思小说中的构思,推进情节发展情节。第二,展现人物关系,塑造典型人物,揭示主旨,表达好构思资料,你的答题要点,是不是这4点?是不是啰啰嗦嗦的重复了,有些又遗漏了。那么我们这6分我们的答题聚焦刚才庙的这几个维度,我们来产生,比如说先我盖住构思对吧?

先由前面的供到后面的读,我们把这个情节全转做一个概述,那么下面对于待遇引发他的生死之悲,他说我一个人寄人篱下,我一个人不在容我苦,不像保胎,有妈妈、有哥哥,不像其他人,总是有一些亲戚,有一些父母,而我是一个孤儿,应该算是一个孤儿,在寄住在贾母外的家里,然后就站在下面就说由共读到转机,它有一个前面我们看是一个媒体,读经是不断的落泪,它的这样一个场景转换,这是一个情节推进。

对于主旨揭示怎么揭示呢?就是最后结果的一个悲剧,亦是在于命运之悲和保待爱情悲剧。当然很多同学可能想不到那么远,那么你只要想一想,能够展现待遇的性情平静,能够展现他们两个人的关系,都可以把这些说到的可以,我们的思考方向是情节推进人物关系的变化,人物性格的解释,主旨精神,我们朝这几个方向来回答。

好,这个是我们提到了构思,那么在上一次高山一轮,包括我们这一次的监控考当中的法学构思起,学生不太会做,所以我们在高三同学有必要梳理一下,构思起。

那么高二同学在学的过程当中,你也关注一下构思及如何体现在文学作品,特别是小说艺术当中。

比如说我们在昨天我在和一位老师备课,他准备上必修下的第六单元的单元梳理课,我们在确定上什么内容,后来就确定准备上第六单元小说的构思艺术。

那么构思艺术它的下位有很多种技巧,比如说我们这里梳理了,我们就举一个大家接触过的文本,比如说最后一行,大家看最后一行,最后一行。线索意向剪裁,其实我还把意向删掉了,包括意向是什么?屋舍跟屋舍相对的印象是房子,你们有没有看这篇文章当中的第一段的第八段?我忘了是第一段的第八段一段的第四行第三段讲建房子,中间有几段第四段冲跑了,第四段到第九段是讲建房,然后房子和屋舍有区别,意向还有整个的文本,这篇散文是以屋舍作为线索来展开来推进的。

再有一个就是选材和剪裁,其中有一个题目就是这篇文章标题是物色为什么文章的中间的三四段写了建房的过程,这就是选材和选材,这些文章里面存过的,那么其实这就是考构思。

再有一个比如说我们给你们印了一篇文本大好时光,刚才我们提到的它是一个人我对话,人物对话其实也是一种构思的艺术。我们刚才说前知必付当中我们是学到过的,包括我们在这一次的我们应高一的监控考,我们考了两篇文章,一篇是赤壁赋,另一篇是舒适的打理端书书,一篇回信,我们就考这两篇文章的构思艺术。

那么有同学高一同学就说,我们不知道什么叫构思艺术,其实在我们高中的教材当中,我们是学过这些的,当然在高考的考试当中,特别是现在的现代王二,他一定会有构思体,我们需要对构思体做一个梳理和细化,每道题我们应该怎么。那么我们刚才提到了构思之妙妙在哪里?如果是小说,它是妙在塑造典型人物、推进情节、展现关系、揭示重审。

如果是散文,如果是写人散文,显然咱们是党员,也是塑造人物,表达情怀,通过场景转换来展现人与人人和物的关系,揭示主旨,主旨就是表达作者的发现,比如说我也去看,我也必谈,在座同学都读过高一同学是高一上的一篇课文,这是一篇散文,对吧?

那么如果说我们从这一类文本它的构思它是用7个章节来创新整合文章,那么每一篇章的构思它都有。

所以我们在阅读时候作息时候我们大部分归个类,比如说我们就读一下,比如说2022年82年春我们都没有做过,他说本来写了树、黄牛、母亲三种货物相关联共享主旨,那么这个考选材和检材,如果说我们就对他做赏析,我首先要讲这三者作者如何把它关联起来,我们一定要做一个叙述。

第二就是这样的系数如何贡献了主旨,主旨是什么?就说题干当中已经限制了,你通过题干我们不出这一题的构思是考考选材和剪裁。

如果说我们读这个药,这个小说它的构思有两个特点,一个是双线,一个是花老少买药,一个是下雨被杀,2条明暗双湖交织,那么还有1个它有4个场景,它有场景转换,通过场景转换时空推进,注意时空推进也是一种构思意识,他来推进情节,来展现人物来交代情形等等。

我们做这个题目大家注意上面就上面那行的后面一个一句话,构思它是手段,它是服务于主旨表达的,那么构思之妙妙在哪里呢?妙在主旨,主旨有哪几个维度?你就朝这几个维度答,这就是我们的产生要点,我们把这个想清楚了。

好,下面我们来说一下整本书的命题,其实现在是一个在探索阶段,各种考法选择题有,主观题有,还有今年的读法也有。那么我想我们在高三包括高二,我们考试的目的主要是让我们不断的重温这篇小说,不断重温,因为内容太多了,很容易忘了。

那么像这这个题目,同学们先读一下最后一行文字,先读一下最后一行的文字。周汝昌的红楼小讲当中他有对于红楼梦的笔法用了两个短语来描述,一叫做一笔多用,第二还有多笔。这里我们也留一个疑问给同学们思考,大家回忆一下第三回宝黛初见。黛玉见宝玉之前,小说用了哪多笔哪几笔来写宝玉?黛玉见宝玉之前有多少铺垫?回忆一下,这就是小说的多比用,所以待遇见宝玉也是似曾相识。

宝玉见黛玉好像我见过这个妹妹,但是前面有很多有人指心说对吧?说保利那边有印象吗?有王夫人说有印象吗?有导有所待遇的妈妈叫农民贾敏,对吧?有这些人物在黛玉的脑中说了说了他一看黛玉好像这个样子,宝玉是不是他们演出这个样子是保持一致的还是形成反差的?

所以宝玉没有出场,每个观众每个读者包括待遇有非常高的好奇和期待,到底保利是个什么样子,那个是多比15年要。

那么现在我们考的是一笔多用之妙,现在我们看一下这一笔,做这个题目题干当中已经告诉你一比多用在哪一叙事,这是我的一个采分点,去了什么事?第二塑造一个怎样的人物?第三,凸显了什么样的主旨?这是你要回答的。

那么这一张PPT我们不讲,因为刚才我们讲过了,反过来讲我们来讲下面的这个是去年高考,因为我们是去年我们的二模我忘了应该是二模卷,所以去年高考三分我也大部分都达到了,他们说在我们复习过了保育挨打。

那么先我们来说一下这一笔,你要概述这一笔,当你考察时候,在座同学们如果现在高一刚刚读了红楼梦的,你马上就说褒义挨打是多因,利果是哪几因?仅仅是不想见甲乙双方,仅仅是甲方告状吗?甲方告状又跟哪个有关系?就是这个东西这一比差不多,不仅仅是看这一回你知道的,你马上就要唤醒你前面的记忆。

首先这一笔一这两分很难得,很多同学达不到这一题到底是一个问题。
说话人4 01:04:34
好。
说话人3 01:04:35
那么多用了用叙事已经叙了,那么还包括哪些叙事情节的推进?环境的展示,同学们在必修上读小说单元,包括必修下,我们有一个单元就是小说就祝福这个单元,有一个教学内容就是人与环境的共生,对吧?环境塑造人人又不停的来塑造环境。

那么我们想宝玉挨打,它其实展现了贾府的一个环境,这个环境是什么?就是教化困境。什么叫教化困境?贾政想教儿子,但是儿子不听,实在没有办法忍无可忍就暴打一顿,暴打儿女是父母亲最大的失败,大家想有用吗?越打越没用,越打越叛逆,越打越反抗,所以母子父子间的博弈,我想不应该是用大的方式,所以这里展现了一种教化困境,教化困境不仅仅是教,教不了正好打,还包括甲午的溺爱,亡夫人的袒护,照顾自己一个改变环境,这是一分。

再往下面走是什么?塑造人物。当然这个人物不仅仅是一个纵深项目,主要人物就是这两次。

那么我们刚才说用一两个词来概括这个人物的特性,父亲是什么?专制,在座的父母亲肯定有专制,我想父母亲管儿子他是一个好心,也是一种责任,但是很容易就专职了。

那么再有一个宝玉这个时候多少岁猜一下。有专门同学家研究过保密,这个时候多大,应该是在年初二年初二年龄初中学生是这样。

当然在大观园当中他写了保密成长的三年,应该是在13岁到16岁当中,这个阶段就是最为叛逆的初二的高一这个年龄段,好塑造人物形象,再有一个就是对情节的展示,对环境和情节融合在一起,它其实是一个荣林二府,还有榆树,为什么有一个甲方告状。
说话人4 01:07:21
对吧?
说话人3 01:07:23
它们之间的一个利益冲突,当然还有一个小说主体的一种预示预示最后贾政管不服儿子。

关于这一段刘在富先生有一段评述,这个评述就是说这就是中国封建的礼俗秩序和追求个性解放自由的矛盾,自由和个性的矛盾,也就成了我们去年澳门的作文题。

当我看这一段的时候,我想就用这个作文。
说话人4 01:08:04
你们可能看过,
说话人3 01:08:09
那么其实在写作文的时候,很多同学都提到了。

你作文题的出处就是我看和在座的黄老师。
说话人4 01:08:18
这本书我的一种。
说话人3 01:08:21
我马上就说因为在进行当中我在想一直有一个作文题的困难,出什么作文题,我想在座的长宁区的教研员这两天正在磨卷子,他也在痛苦,用什么作文题来考你。

昨天晚上我也在帮黄浦区省跨国体系作文题我也就很满意,但是在打磨,就是说那么我们想一个作文题出来,它一定是揭示很多共性的问题。共性的现象。再有一点就是刚才我提到了,宝玉这一挨打,其实对于保利来说是好事还是坏事?两大好事。虽然身体上打的爬不起,屁股打肿了,但是因为有甲午的袒护,他从此可以上去整天的游手套写完了。

第二,到了晚上到了下午黄昏时候,黛玉来看,他生了就怕黑黛玉从此他们俩的关系进入了一个稳定期,心心相印,所以我们说祸福相依,那么这一段这个情节我们要讲的是一笔多用,我们再强调多用,你朝哪几点维度来答题。我们在答题当中我们特别不断的强调几个要点,一个这一笔的修饰对象,这一笔的作用之妙的哪几个维度。
说话人4 01:10:03
这是我们。
说话人3 01:10:04
的哪里当中的一个核心要义。如果你想清楚这些问题,你的答题。
说话人4 01:10:08
思路是非常清楚的。
说话人3 01:10:09
花很少时间得到最好的分数,如果说你到二模前你答题都是传教,你搞不清楚。

命题在问什么,我怎么思考这个答案,我怎么朝哪几个方向咱们都不清楚。
说话人4 01:10:28
就是。
说话人3 01:10:28
写满了我行那就完了,花了时间不得丰富,不得要。

我们也不展开。

我只说一下,这是贾府肾衰事件的重要事件大事件。

那么这个事件如果我们要读,你一定要读清楚朝鲜大官营的众生相,他塑造了非常典型的人物,他也多用称要用在哪几,一个是人物,一个是核心是荣国府和宁国府之间的矛盾,还有一个就是看出那句话,贾府从此最后一个叫树倒,不再看一下,拉开了序幕,这个序幕就是大观园的小姑娘们死的死,出家的出家,出嫁的出嫁,结果就是这样的。

你们可以看一下第七十四回抄检大观园以后,大观园完全没有了当年写诗歌的那种怎么样?青春样态完全是两两种感觉。好,我们这里不展开,因为这个和前面那个题目是一个开放题,如果这三回你记不住,你就写你记得住的。我们一般记得考的都是重大事件,重要人物是两条线,保代勤线之间贾府盛衰事件的重大事件。我们富硒也是聚焦这一点。如果我们说下题外话,其实今年我们中考的备用卷考的就是用于摘要,这个题目是我们去年三模的第七小题,这个题目有没有同学用来写青年的冲考题?

今年中考题质检的时候,我看到同学写这篇文章的素材,重要的东西在很早的时候被我们提到过。
说话人4 01:12:48
钱学森之问。
说话人3 01:12:50
李先生之问都是重要的东西,很早就已经提到过了,那么在今天它有现实意义吗?是一个非常好的写作素材,同学们在做模拟题的时候不要忽略掉。你要我们讲一机两师一心多用用,就是说你说没有写作素材,其实每天我们都在看手机。

好,我们题外话回到题目,题目当中的题干第一句话是论文摘要的我们的答题的限定,我们要求一写作研究目的一分,第二研究方法路径,在这里是两份,结论是两分,是这样来给分的,当然研究方法达到打满了2分,研究结论打满了2分,给满4分为止。

这个题目其实5分6分更好区分,当然因为整张卷子它有一个分数的把控,所以我们就给了4分,我认为给6分更好,区分度一定很好。

那么这个题目我们来看一下你们的第一个问题,整个李月生去问问的是为什么科学和工业革命没有在近代中国发生?

我们经常说当我们读唐宋历史文化的时候,特别是宋代的经济文化,宋代的时代的中国就是现在的美国是一个霸主地位,不管是经济还是文化都是一个霸主地位,那么为什么到了清末就往下走呢?当然现在我们又开始在复兴对吧?走进新时代的复兴,为什么我们走向这样一个低谷?那么吕跃胜就提了这样一个问题,当然这个问题不是中国人提出来的,好,那么这是一个课题。

这是我从浙江大学它的一个校刊上面找了一个课题,那么它们是什么?我们先来看一下研究方法,它有两个是一个理论假设,一分,第二是检视历代的书目,就是你们的表格。这是他的路径和方法,又是一分。

结论各有两个结论,第一个结论就是推出儒家作为单一全部的主流意识形态是主导因素。我们在读历史,罢黜百家,独尊儒术,就发现儒家思想作为一个主营思想统治了中国几千年,其实今天依然我认为依然还是中国人的卷,说得好听是入市是向上对吧?如果说你躺平摆烂,那么很多人就说这样是不行的,这个社会怎么能进步对吧?所以这个是两个问题。那么第二个两个前面是理论,后面是实证,就是历代的书目,通过数据分析,数据分析最有信度最有说服力,来印证理论假设。

那么我们可以看给分,第一个一分两后面两个两分,后面两个是达到后面第二点的两分,达到再一点的给满分,这个研究有一个衍生出的结论,这个结论不作为我们的采分点,如果说你要写上也可以,但是如果前面你没有打码的,你破前后面走,我们可以给你充一分也可以作为一个补充答案,那么其实这个作业出现在你们的选b下第四单元的单元任务,我不太清楚你们当时有没有训练过?没有关系,没有训练过,我们现在训练也来得及。

因为在座的读大学写毕业论文,你就要写论文摘要。

好。
说话人1 01:17:12
这个是我们。
说话人3 01:17:17
上一周一高一监控考试,我昨天看了顾老师发给我的作业,我觉得写的挺好,所以我们就没有必要展开,这个题目是我们有限定,在必修上高一后面的同学必须上,第三单元,我们有一个单元的写作任务就是写文学馆,可能那么核心转型有三个要义,第一是选择角度对吧?第二就是对于你的选择角度,你把你第二就是深入的研究读懂作品。

第三表达方式是虚拟结合,先叙述你的评价角度,再对进行评议,站在一个阅读者的立场来对作品进行评价,那么这个习惯我们为了更好评分,我不让你选角度给你角度,从意向选用的角度,因为必修上第一单元,必修上第三单元都是有关于驿站文学知识的一个把握。

那么现在关键是我们写短评是说诗人用了哪些意象,它汇成了一个什么样的画面?他营造了什么样的意境?他传达了什么样的情怀?我们就把这个要点做出来或者利用意向的效果。

那么在在写这一点上,有同学问他和鉴赏区别在哪里?鉴赏题赏是说它的好,那么如果是短评,你既可以说它好,你也可以说它的短板哪些地方写的不好的,完全可以。所以短评的评价更开放,我们在暑假时候可以补充学生的很多合理的理解准确的答案。那么像这里我们可以看一下,这是一首写景诗,那么你在读的时候,你先把意向提炼出来。第二,这些意向它描绘了一个什么季节、什么地点、什么特征的画面,我刚才问了三个问题,什么季节,什么地点、什么特征?传达作者什么情感,4个要点就出来了。

还有一个问题是作者不是农民,作者的身份在注释当中给你有提示,大家可以看一下作者他其实是一个相当于地方的什么?这个官员他是管理是一个什么?是一个管理部队的官员,所以他在表达情感的时候,应该是传递出分享农家秋收的喜悦。

不是他种的地,他只是起码看到秋收的场景,秋收的田野,那么还有一个他抒发他一种享受,就是热爱田园的情怀,那么有所以这6分当中我认为意向的是必须要有的。
说话人2 01:21:26
什么季节的。
说话人3 01:21:28
什么地方的,什么特征的景,这个是必须要有的。

至于传达的情感,我认为可以补充一些,但主要的都是一些比较好的,比如说的喜悦享受这些情怀。这一题其实我们是把它和成本就进行阅读。

那么我想这个题目我们去年批的时候,谁的文字答的比较好,答的比较好是首先对于什么?对于教化权力的理解,在上一周我们考了高一监控考,其中有一个问题是法治社会还需要教化吗?这个题目原则0分,原因是很多同学不知道什么叫教化,就说没有掌握这本书当中的核心概念。

同学们进入学校以后,每天都在接受教化,每天都在接受教化,教化的过程就是形成文化自觉,养成国家文化认同。

所以我们说比如说像香港回归这么多年以来,他们不用我们的教材,所以才会有97年的一系列的事件,你看澳门就不一样,澳门是用我们的普遍教材,那么你的教材不同,其实教材的学习就是我们接受国家意识国家认同的过程,你才会认为你是中国人。

那么回到这个题目,我们要评价的是小慧的结论,小慧是谨守。我只听了就能够修身立士。

那么有一个前提就是先贤的遗训和长辈的教诲,是不是第。
说话人4 01:23:36
一个。
说话人3 01:23:38
都符合今天的主流价值观?

第二点,不管是先贤还是长辈长辈的教诲,他们是否能够理解当下的树立时代的文化和科技,未必,其实现在年轻人他们在很多方面远远超于他的长辈,因为我们在座的其实我也有很多的话,年轻人每天在学习,但是我们作为父母却每天晚上回家以后能看电视,你叫你的儿女好好读书,他说你在干嘛?你不是也在看电视休息吗?也就是说年轻人在进步,长辈却在原地踏步,甚至在退化,你说这样可以吗?

所以我们在回答这个问题的时候,应该是基于当下的时代语境来回应长辈的教诲,我们如何来用发展观来接纳,来改进来完善,所以这个问题应该是有一个时代引进,用发展观就是我们选b中第一单元,有一个毛泽东日报,有一篇光明日报的什么文章,实践检验真理的文艺标准,它是用发展观来看问题来做评价。

好,这个我们不说了,这是你们中考的19小题写人物短评,因为时间关系我们来讲一下这一题。我不知道同学们高三同学做这个题目以后,你们有没有判断这个题目考什么?它和我们以往考小选的不同在哪里呢?同学们看你们的讲义,讲义当中我们的命题依据,第一个就是课程标准,课程标准当中对转书的阅读有一个阅读这个学术论著,建构论著的阅读经验和方法,阅读召回小说,建构长篇小说的经验和方法。

也就是说我们读整本书不仅要读内容,还要读还要建构积累阅读方法,关键是阅读方法怎么考呢?

就这么考,今年就考了阅读方法,所以我们在讲读法的时候,大家看一下我最上面那行字,我们要掌握不同文类的读法,依照文体类型,依照着作类型来阅读。比如说章回小说,我们依照章回小说回目的叙事评价功能来梳理阅读。我们读学术论著关注核心概念和主要观点,梳理他提出问题解决问题的思路。我们是这样来读的,两种不同背景的书的读法不一样的,所以在这里我们不是讲答案,而是讲可能我们必须要掌握不同文类的读法。

那么读法在考试当中,命题人是依照文类来命题的,好,阅读者是依照文类来阅读的,所以比如说在这个题当中的第二点,理解作者对重视契约的现代社会,重视规矩的乡土社会的比较,也就是说我们读乡土中国的后的后继,你们还想得起来吗?费孝通说我运用的研究方法是社区研究的比较方法,就在西方社会的参照下,我们来研究上海。好,最后我们做一个小结,是生育,我们今天讲的生物应用型的配套,第一点是他们是高三同学,要梳理公共的高中5本教学当中的学科知识,刚才我们主要刺激术,其实乡绅这样的例子。那么在了解命题当中的命题能给定的任务的限定,运用知识来解决问题,所以现在的考试叫学以致用。
说话人4 01:28:47
学以致用。
说话人3 01:28:48
的前提是你要掌握学习的学科知识,你才能用。

第二点就是我们刚才提到的。当下的考试板块有社科文、有小说,有散文时报,有诗歌,我们要掌握这些文体的知识,还有乡土中红楼梦的主要内容。第三就是掌握不同文类不同著作的读法。今天我们讲的综合利用题有答题当中有三个要素,第一点就是很多中国应用型是开放性的,来他问你是否你首先要表明观点的鲜明性,第二点就是限制性。你答题的时候你要清楚命题人给定的材料。
说话人4 01:29:42
有哪些。
说话人3 01:29:44
命题的给的任务有哪些,要清楚它的现状的命题指向。

第三就是我们答题,有观点有理由做到逻辑严谨,我答题的完整性,这是综合应用题要求的三点开放限制。完整。

最后一张是我梳理了不同文类的阅读的方法,我想这个不展开,那么我们同学们现在要做的是备考当中的这一行。
说话人5 01:30:22
夯实知识。
说话人3 01:30:23
建构读法或者叫学法。你们的考试练习就是学以致用。

下面这些文类左边的是他的艺术魔法,右边的他写的目的,所以意图、效果、作用、妙用都指向他的写作目的。这个脑子清要清楚手段和目的的关联。最后我们用胡适他几十年前在北大讲座送给毕业生的英语,也是佛经当地叫功不唐捐,你花了功夫一定不会白白浪费,用这句话来共勉各位,后面包括我自己。好,今天我们就到这里结束。
说话人5 01:31:32
对,这样好吧。
说话人1 01:31:33
有问题的同学可以留下。





\end{document}