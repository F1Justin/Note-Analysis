\documentclass{ctexart}
\usepackage{geometry}
\usepackage{fontspec}
\usepackage{tikz}
\usepackage{hyperref}  
\usepackage{CJKfntef}

\hypersetup{hidelinks,
	colorlinks=true,
	allcolors=black,
	pdfstartview=Fit,
	breaklinks=true
}

\newcommand{\mybox}[1]{\tikz[baseline=(MeNode.base)]{\node[rounded corners, fill=gray!20](MeNode){#1};}}
\geometry{papersize={21cm,29.7cm}}
\geometry{left = 2.8cm, right = 2cm, top = 2cm, bottom = 3cm}
\pagestyle{headings}
\setCJKmonofont{LXGW WenKai Mono}
\ctexset {
   abstractname = {本文概要},
   today = big,
   section/name = {第,节},
}

\newcommand{\df}{\large \mybox}
\newcommand{\nm}{\normalsize}
\newcommand{\blk}{\vspace*{1\baselineskip} }
\newcommand{\blkz}{\vspace*{2\baselineskip} }
\newcommand{\blkx}{\vspace*{4\baselineskip} }
\newcommand{\blkc}{\vspace*{6\baselineskip} }
\newcommand{\blkd}{\vspace*{10\baselineskip} }
\newcommand{\blkv}{\vspace*{16\baselineskip} }
\newcommand{\blkb}{\vspace*{32\baselineskip} }
\renewcommand{\\}{\par}

\setcounter{secnumdepth}{4}
\setcounter{tocdepth}{2}

\begin{document}


\title{语文分析}
\author{F1}
\maketitle
\linespread{1.48}

\tableofcontents
\newpage
%『』
\part{答题模板}

\section{叙述,描写与抒情}

\subsection{『比喻』修辞之比喻}
比喻是常用的修辞手法.\\
\df{比喻/比拟?}\\
\df{描绘了什么样的画面?}\\
\df{能怎样表达主旨?}\\
\subsubsection{例题}
请分析第二段划线句在文中的作用.\\
\texttt{句中将人,云,山分别比作浮萍,睡莲,微生物,吐出了人与自然事物在宇宙背景下的渺小,比喻大胆新奇,吸引读者阅读兴趣.为后文描写雨,云,水做铺垫,奠定圣洁高远的基调,反映了作者的崇敬之情.}
\subsubsection{例题}
第三段划线句很有表现力,请加以赏析.\\
\texttt{本句运用比拟手法,赋予水银以人的情态,"欢欣鼓舞","朝我眨眼",生动表现出水银泄地后程珠状扩散之快与我在黑夜中见此情景的以外之喜.又运用比喻手法,将泄地的水银比作满天繁星,不仅生动渲染了银珠之多,还巧妙地将天地空间倒置,与下文的时间回溯和谐地融为一体.}\\
\nm \fangsong
\centerline{风吹云动}\\
\centerline{傅菲}\\
\\天空一无所有
\\何以给我安慰
\\这是海子的诗句。其实天空有云。云也只游荡在天空里。天空是云的居所。
\\云可能是最轻的东西了,它终生被风吹动。风托着它,拽着它,改变它的形状。风让云聚成一团,也让云成流丝。山区多云,也多风。荣华山的上空,盘踞着云,满池塘浮萍似的,让人卑微:人只是池塘里的微生物,荣华山也只是一朵水莲。
\\荣华山草木葱茏,水蒸发量大,多云是惯常的。云带来了雨。或者说,云是雨的前世,雨是云的凡胎。凡胎注定在大地上浪迹。
\\初入荣华山,是夏季。炽热炎炎。我一下子注意到了云。云白如洗,蚕丝一样。天空蓝,蓝得没有尽头,蓝得深邃无穷。我对本地人疑惑地说:这天蓝得只剩下云的白了,过滤了一样。本地人望望天,说:云黑起来才吓人呢,像藏着恶魔。
\\四个月后,我见识了恶魔一样的云。白露没过几天,气温急剧下降。午后,天完全黑暗,山下盆地像个地窖。蚂蚁慌乱。院子里来了很多蜻蜓,四处飞。天是在十几分钟内暗下来的,空气如洇开了墨水。我关掉电闸,收拾翻晒的物什,坐在走廊。云乌黑黑,一层层压实铺开。云团山峦一样,一座连一座,形成绵绵群山。高耸陡峭。云团不移动,遮蔽了光,给人压迫感。
\\游动的光,蓝色,在云层突闪,爆出蜘蛛丝一样的裂缝。闪电来了。我们不叫闪电,叫忽显。忽然显现的光,照见了云团狰狞的面目。云团像戴着黑色面具,披头散发的傩舞人。雷声从天边轰轰轰传来,俯冲而下,隆隆隆隆,炸裂。闪电一道追一道,显得迫不及待。蓝色火焰啪啪啪瞬间熄灭。似乎它快速地到来,是为了快速地熄灭。云团被一层层炸开。
\\雨下了。豆珠一样,啪哒啪哒,急急地敲打地面,溅起干燥的灰尘。脆脆的雨声,犀利。雨珠打在白菜上,菜叶弹起来。雨点密集起来,雨线直拉拉。雨线网住了视野。鱼从水塘跃起。蝉声消失耳际。芭蕉花一朵朵打落在地。天慢慢白,把暗黑色一层层蜕下来,露出水光色。
\\杂工老钟每天出门,戴一顶旧得发黄的草帽,帽檐低低。他望望天,说:今天没有雨。或者说:云压头上了,有大雨。他把草帽当扇子,边摇边说话。他也把草帽当坐垫,草帽往屁股下一塞,摸出烟,说:这个天会不会热死人啊。汗滴在眉毛上了,抬手用衣袖擦。衣袖两边结了很多盐花。他是靠天吃饭的人。挖菜地劈木柴遮秧苗,都是他的活。起风了,我站在窗口,喊:疤脸,疤脸,看看云,会不会下雨啊。疤脸是老钟绰号。他喜滋滋地翘着烟,说:这个天下不了雨,别看云那么厚,风吹吹便没了。
\\山里的人,都会观云识天气。挖地,挖了一半,把锄头扛在肩上回家了。问他:怎么不挖完就回去了。他嘟嘟嘴巴,说,你也不看看云,暴雨马上来。云团还在天边呢!这里艳阳当空。可隔不了一碗茶时间,乌云盖顶,噼噼啪啪,暴雨来了。
\\云怎么也散不去,厚厚的,一堆叠一堆。云是最高、体量最大的山峦。山峦慢慢塌,以暴雨的方式坍塌。荣华山四周的盆地成了云山的地下河。云带来了充沛的雨量。手抓一把土,水飚射。辣椒烂根在地里。昨夜刚开的蔷薇,被无情地摧残。南浦溪的木船不知道漂到哪里去了。瓦漏了,哗哗哗的雨水落在了锅里,落在木板床上。过河的山麂溺水而死。
\\最彻底的洗礼。云再一次把大地恢复了原始的模样。摧枯拉朽是最彻底的清洗。云有一双魔手,让即将死亡的加速死亡,让无力生存的加速腐烂,让散叶开花的尽快茁壮成长。腐朽的,僵硬的,都埋到泥浆里去吧。
\\云随时随刻都有一种寄情穹宇的状态。像一个不问人间的隐居者。王维有诗《终南别业》:
\\中岁颇好道,晚家南山陲。
\\兴来每独往,胜事空自知。
\\行到水穷处,坐看云起时。
\\偶然值林叟,谈笑无还期。
\\散步到流水尽头,云正好从山头涌上来。其实流水没有尽头,尽头之处是终结之处。在乡村寿枋(棺材的别称)有一副常见的对联:水流归大海,月落不离天。万事万物,都遵守恒定律。人需要从容生活,淡定,淡雅,淡泊。王维被尊为诗佛,他了悟水云之禅。他四十开始,半官半隐,在陕西蓝田的辋川寄情山水。辋川青山逶迤,峰峦叠嶂,幽谷流瀑布,溪流潺湲。我想起自己不惑之年,仍在外奔波,让家人牵挂,多多少少有些悲伤。
\\陈眉公辑录《小窗幽记》,引用洪应明的对联:
\\宠辱不惊,闲看庭前花开花落
\\去留无意,漫随天外云卷云舒
\\云自卷自舒。又几人可以卷舒呢。人永远没有满足的时候,人永远不会珍惜已拥有的。水到了大海,才知道,哦,所有的行程只为了奔赴大海。
\\窗外,是晚霞映照的山峰。入秋的风,一天比一天凉。干燥的空气和干燥的蝉声,加深了黄昏的荒凉。夕阳的余光给大地抹了一层灰色。云白如翳。一个穿深色蓝衫的人,坐在溪边的石墩上画油画。他每天都来,坐在同一个石墩,已经有半个月了。我偶尔去看看他画画。他画田畴,画山梁,画云。云像什么,我们便会想什么。云,是心灵绽放出来的花。云是云,我们是我们。云不是云,我们不是我们。云是浮萍,我们是微生物。
\blkx
\\\centerline{水银花开的夜晚}
\\\centerline{迟子建}
\\有一日傍晚咳嗽流涕,我便取放在玄关托盘上的体温计,想看看自己是否发烧。
\\我取体温计的时候,不慎将外壳的护帽朝下,这一竖不要紧,由于对接处咬合不严,护帽叛徒似的落地而逃,将体温计彻底出卖了,它随之坠落,摔成两截。
\\它这一跌,我家的黑夜亮了。
\\从玻璃管内径流溢而出的水银,魔术般地分裂成大大小小的珍珠状颗粒,像一带雪山巍峨地屹立在我面前。我先是拿来一块抹布擦拭,以为它们会像水滴一样,迅速被吸附,岂料它们欢欣鼓舞地一分二、二分三、三分四地遍撒银珠,泻地水银非但未少,反而如满天繁星,在白桦木地板上,朝我眨眼。它们近在咫尺,却仿佛远在天边,不可征服。
\\我少时对水银的了解,竟来自当时广为流传的一本小人书《一块银元》,主要情节围绕一块银元展开,写了穷人的苦,地主的恶,其中最让人惊悚的情节,是一个地主婆死了,她的儿子竟让一对童男童女为他老娘殉葬。他们给童男童女灌注了水银。故事浓墨重彩的是那个身世凄惨的童女,在出殡的行列中,她端坐在莲花上,手持一盏纱灯,双目圆睁,虽死犹生。她的亲人在路旁声声唤她,可她无法应答了。那个画面给我幼小的心灵,带来了浓重的阴影,恨地主,也恨水银。水银是毒蛇,它要了如花似玉的姑娘的命!
\\我那时感冒了,发烧了,抗拒去卫生所,骨子里是恐惧水银体温计。总觉得我的腋窝藏着火苗,会将爆竹似的它引爆。它灿烂了,我就黑暗了。体温计是恶魔,这在看过 《一块银元》 小人书的同学心中,根深蒂固。以至于我们憎恨一位班主任老师时,私下议论要是小人书中被灌注了水银的是她,而不是那个女孩,该有多好。这位班主任是我们的语文老师,她中等个,微胖,圆脸上生满雀斑,厚眼皮,眼睛不大,但很犀利。
\\我们为什么怕这位老师呢? 她严厉起来不可理喻。她有一杆长长的教鞭,别的老师的教鞭只在黑板上跳舞,她的教鞭常打在学生手上。期中期末考试总成绩不及格者,是她惯常教训的对象。她会让他们伸出手来,这时她的教鞭就是皮鞭了,抽向落后生。痛和屈辱,让被打的同学哇哇大哭。这种示众的效果,倒是让所有的学生不甘落后,刻苦学习了。但大家心底对她还是恨的,她头发浓密,梳着两条粗短的辫子,我们背地就说她带着两把锅刷。
\\最让我们难堪的是检查个人卫生,我们上课前她会手持碎砖头,高傲地站在门口,我们则像乞丐一样朝她伸出手去,如果我们的手皴了,或是指甲里藏污纳垢,她会扔给你一块碎砖头,让我们出去蹭掉手上的皴,抠出指甲里的泥,砖头在此时就成了肥皂了。
\\这位班主任老师看上去跋扈,但她业务好,很敬业,也有善心。有的同学家贫,她家访时会带上她买的作业本,她还帮助交不起学费的学生交费,并带我们进城,去照相馆拍合影。如果是冬天,天黑得早,讲台就点起一根蜡烛。烛火跳跃着,忽明忽暗,她的脸也忽明忽暗,那也是她最美的时刻。她不用教鞭,脸上的雀斑看不见了,语气温柔,面目平和。
\\她离开我们小镇,似乎没有任何预兆。突然有一天,她要调到她恋人那儿,是去结婚。这时我们才意识到她是一个女人,是个有人惦念的人。
\\她要离开了,按理说我们是该同声庆祝的,可大家突然都很沮丧。她将自己所用之物,分给常遭她鞭打的人,那多是家庭困难的同学,我听说的就有书本、衣物、脸盆。在她走前,有天我在小卖店碰见她,她还买了一双雨靴送我。从此后她离开的风雨时刻,穿着雨靴走在泥水纵横的小路上,总会想起她。而她带我们拍的合影,成了同学们最美的珍藏。
\\四十多年了,我没有她的任何消息,也极少想起她来。但水银泄地的这个夜晚,也过了半百之岁的我,却很热切地思念起她来。不知她是否还在她当年嫁过去的小城。按她的年龄,应是儿孙满堂,颐养天年了。
\\夜一点点地黑起来,我清理完地板上的水银,关了厅里的灯,打算回卧室休息一下。借着卧室的微光,我突然发现刚清理过的地板上,仍有水银珠一闪一闪的。我不相信,取了手电筒照向那里。呵呀,这分明是一个微观花园么,我发现了无数颗更加细小的水银珠粒,在白桦木地板的表面和缝隙,花儿一样绽放着。
\\这不死的花朵,实难相送,那就索性不送,我不相信就凭它们,会让我性命堪忧———将其当花来赏又如何! 权当它们是腊梅的心,是芍药的眼,是丁香的小袄,是莲花的罗裙!
\\因为在黑夜面前,所有的花朵都是无辜的。


\subsection{『段语』``这段话在文中的作用是?''}
对于提问一段话在文中的作用的题目,要注意以下几个点:\par
\df{承接上文内容概括}, \\
\df{启接下文内容概括}, \\
\df{这段话的概括}, \\
\df{这段话的文学性}\nm (如果有的话),\\
 \df{对全文情感主旨的作用}\nm 比如:凸显.

\subsection{『风格』本诗/词的风格是...}
诗词风格多种多样, 但是如果按照抒情方式分类主要有\par
\df{委婉}\nm ---寓情于景和\par
\df{豪放}\nm ---直抒胸臆\large \par
两类.

\subsection{『群像』塑造众多人物形象的作用是...}
人物群像的描写重在由共性与个性的对比中突出\textbf{氛围}.\\
\df{共性}\\
\df{个性}\\
\df{氛围}\\
\large
\subsubsection{例题}
与聚焦单个人物形象不同,本文塑造了众多人物形象,分析这种写法的作用.\\
\texttt{本文描写了多个典型人物,如勤恳工作热于助人急性子的阿德;善于沟通引导病人的童医生;走到哪都不忘交易的淳朴的村妇们;笨拙又清闲的书贩.共同描写出乡村淳朴热闹的氛围,比起单独人物描写更有利于表现勤劳可爱的劳动人民群像,体现了作者的赞美之情.}\\
\nm \fangsong
\\\centerline{被劝进来的病人}   \\\centerline{干亚群}
\\逢三与逢七,是小镇的市日,类似于我老家的赶集。碰上市日,医院比较忙,病人把集赶了,顺带把自己的病也看了。
\\市日把村民赶到镇上。毫无遮拦的市场里,村民们挑肥拣瘦,掂斤捻两,最后以惊人的耐心杀价掐价。市场上的果蔬大多是自产自销,所以,他们买卖人的身份一个月里经常在换,轮到别人向自己砍价时,嘴上吵吵嚷嚷,手上却不让人吃亏,秤早已捏了起来,秤尾往上一翘,顾客的头随之一歪,一桩生意就完成了。
\\太阳跳上树梢,把市场照得像块煎饼时,人们各自完成买与卖,然后周围的声音慢慢浅下去,摊位上的东西也渐渐薄起来,零乱的脚印,散落的垃圾,以及花花绿绿的鸡屎,跟灵感跑了一半的油画似的。
\\市日把一撮人劝进了医院。他们带着集市的痕迹来看病。他们把拖拉机的突突声拐进了医院的大门,手推车咕噜咕噜,一个侧身依在墙角,自行车前架后搁,心事重重似的靠过来,医院的天井一点点被它们拥塞。清洁工阿德挥舞着扫帚,指挥着拖拉机停这边,手推车放那边,至于自行车,一律摆到车棚。容不得商量。一旦有人把车放错了位置,阿德就提着扫帚跑过去,如果来人不配合,阿德的脸就开始涨红,话也结巴,脖子上青筋凸起。如果有人来看病找不到医生,他会满医院地帮忙去找,一边找,一边大声咳咳,似乎在打暗号。  
\\到了医院,买卖人变成了病人。对他们而言,医院跟集市无非是换了个场景,仍用刚才吵吵嚷嚷的声音陈述自己身上的某个痛点。医生当然不会仅限于病人一句肚痛头晕就开方子,肯定要问清肚痛的来龙去脉、前因诱因。而病人翻来覆去跟烙饼似的停留在自己的痛点上,医生需要的信息仍云遮雾绕。
\\我坐在童医生对面,彼此是同事,但在和病人打交道这件事上,她是我老师。吵吵嚷嚷的声音里,童医生看上去很惬意,看见病人既不问病史,也不做检查,而是先笑嘻嘻地问病人今天市日又买了啥,然后夸病人会买东西,价格实惠。病人听了,似乎觉得自己捡了一个大便宜,语气开始亲切起来,甚至掀开篮子给童医生看自己买的东西,童医生侧过身,极认真地看了看病人的篮子,再次夸病人会买东西。俩人像是街头偶遇的老朋友拉起了家常,饮食咸淡,起居习惯,聊天把买卖人劝进病人角色。他们一股脑儿地把自己最近的生活史复习了一遍。就在病人絮絮叨叨时,医生的问话戛然而止,一张处方已递到病人面前,仿佛是市日里的一杆秤。
\\病人一坐到我前面,我根本不会像童医生那样转弯抹角地先跟病人温习市日,而是直截了当地开启病人与医生的模式。他们的病痛大多是积累起来的,问他们为什么不早点来看,回答几乎是一模一样,等市日时来看,似乎特意来看病是一件难为情的事。    
\\他们看过内科看外科,看过外科看牙科,一次次来到医生面前。而她们,闪进了右侧的诊室。她们进来时不像是看病,倒像探病,一身花衣服。她们手里提着七七八八的东西,声音也是七七八八,似乎集市的热闹仍然悬在舌头上。有时等我给病人做好检查出来时,发现突然少了几个人,原来是跑到院子里做生意去了。她买她的花裤子,她买她的红番茄,然后,俩人你提着我的花裤子,我拎着你的红番茄,再次进入诊室,脸上荡漾着番茄红。还好,她俩的病情不一样,否则我真怀疑她们刚才把病也交易了。
\\市日上的事,像边角余料似的被病人带进了医院。有人说有一个老头,每次市日摆旧书摊,可等他把书摆好,市日就散了,于是他又把书一本本收起来,几乎没有做过一笔生意,看上去像来晒书的。我置身在他们的闲谈中,忍不住问,他是卖的,还是租的?说话的人摇摇头,然后一屁股坐到童医生那儿,似乎把老人旧书摊这件事压了下去。
\\虽然市日是医院看病最忙的日子,但病人看病的时间都不长,大多病人出去时手里只不过多了一张方子,有的甚至方子都没有。到了十点半后,重新空荡荡的,却留下了一堆堆的花花绿绿,已经分不清是鸡屎盖着鸭屎,还是鸭屎压着鹅屎,唯一可以辨别的是羊粪,院长戏称是“六味地黄丸”。
\\阿德站在院子里咳咳咳。不一会儿,大家从科室里出来,脖子上挂着听诊器,而手里提着扫帚、冲水器,听从阿德的指挥,开始清扫院子,仿佛走的是客人。



\subsection{『人形』请评点此处人物形象的描写}
对于人物形象的问题,需要注意\\
\df{神态变化}\\
\df{情感变化}\\
\df{体现了什么}\\
\subsubsection{例题}
\large 
从人物描写的角度,为甲段划线句写一段评点文字\\
\texttt{划线句使用了肖像描写和语言描写,连用动词"一怔","睁开","咧开嘴"等,生动地描绘了女人见到水生后由惊愕到欣喜再到数年以来委屈的悲涌上心头的心理变化,体现了女人对水生的思念之情感之深,反映出过往战争的日子的艰难.}\\
\nm \fangsong \\\begin{center}嘱咐\end{center}\\\begin{center}孙犁\end{center}\\太阳平西的时候,水生望着树林的疏密,辨别自家的村庄,他的家就在白洋淀边上。家近了,就要进家了!他想着许多事,父亲是不是还活着?父亲很早就有痰喘病;还有自己的女人,一别八年,分别时她肚子里正有了孩子,是不是都活着?房子被烧了吗?\\他在院子门口遇见了自己的女人。她正悄悄地关闭那外面的梢门。水生叫了一声:“你!”\\女人一怔,睁开大眼睛,咧开嘴笑了笑,就转过身子抽抽打打地哭了〔甲〕。水生看见她脚上那白布封鞋,就知道父亲准是不在了。两个人愣在那里站了一会。还是水生把门掩好,说:“不要哭了,家去吧!”他在前面走,女人在后面跟, 走到院里,女人紧走两步赶在前面,到屋里去点灯。\\他走进屋里,女人从炕上拖起一个孩子来,含泪笑着说:“来!这就是你爹, 一天价看见人家有爹,自己没爹,这不回来了。”说着已经不成声音。\\水生说:“来!我抱抱。”那孩子从睡梦里醒来,好奇地看着这个生人。 水生在黑影里问:“你叫什么?”“小平。”“几岁了?”女人转身插好门,对孩子说:“别告诉他,他不记的吗?”\\水生看着女人。离别了八年,她并没有老多少,头发虽然乱,脸孔苍白了一些,可那两只眼睛里的光,还是那么强烈。\\
女人歪在炕上,笑着问:“说真的,这八九年,你想起过我吗?” “想过。”“怎么想法?”她逼着问。\\“临过平汉路的那天夜里,我宿在一家小店,小店里有个鱼贩子是咱们乡亲。我买了一包小鱼下饭,吃着那鱼,就想起了你。”\\“胡说。还有吗?”“没有了。你知道我是出门打仗去了,不是专门想你去了。”\\“我们可常常想你,黑夜白日。”她突然支着身子坐起来,问:“你能在家住几天?”\\“就这一晚上。我是请假绕道来看你的。”“为什么不早些说?”“还没顾着啊!”\\女人呆了。她低下头去,无力地仄在炕上。过了半天,她说:“那么就赶快休息吧,明天我撑着冰床子去送你。”〔乙〕
\\鸡叫三遍,女人就起来给水生做了饭吃。这是一个大雾天,地上堆满了霜雪。女人把孩子叫醒,穿得暖暖的,背上冰床,锁了梢门,送丈夫上路。出了村,她要丈夫到爹的坟上去看看。水生说以后回来再去,女人坚持要去。她说:\\“爹叫你出去打仗了,是他一个老人家照顾了全家。这是什么日子呀?整天价东逃西窜。你不在家,爹对我们娘俩的照顾,只怕一差二错,对不起在外抗日的儿子。夜里一有风声,他就把我们叫醒。他老人家背着孩子逃跑,累的痰喘咳嗽。这些个担惊受怕的日子,把他老人家累死。还有那年大饥荒……”
\\在河边,他们上了冰床。水生坐上去,抱着孩子,用大衣给她包好脚。女人站在床子后尾,撑起了竿。女人是撑冰床的好手,她逗着孩子说:“看你爹没出息,当了八年八路军,还得叫我撑冰床子送他!”\\她轻轻地跳上冰床子后尾,像一只雨后的蜻蜓爬上草叶。轻轻用竿子向后一点,冰床子前进了。大雾笼罩着水淀,只有眼前几丈远的冰道可以望见。河两岸残留的芦苇上的霜花飒飒飘落,衣服上立时变成银白色。她用一块长的黑布紧紧把头发包住,脸冻得通红,嘴里却冒着热气。她连撑几竿,然后直起身子来,向水生一笑。小小的冰床像离开了强弩的箭,摧起的冰屑,在它前面打起团团的旋花。前面有一条窄窄的水沟,水在冰缝里汹汹地流,她只说了一声“小心”,两脚轻轻地一用劲,冰床就像受了惊的小蛇一样,抬起头来,窜过去了。\\水生提醒她说:“你慢一些,疯了?”女人擦一擦脸上的冰雪和汗,笑着说: “同志!我送你到战场上去呀,你倒说慢一些!”\\“擦破了鼻子就不闹了。”“不会。这是从小玩熟了的东西,今天更不会。在这八年里面,你知道我用这床子,送过多少次八路军?”\\冰床在霜雾里飞行。“你把我送到丁家坞,”水生说,“到那里,我就可以找到队伍了。”\\女人呆望着丈夫。停了一会,才说:“你知道,我现在心里很乱。八年才见到你,你只在家呆了不到多半夜的工夫。我为什么撑的这么块?为什么着急把你送到战场上去?我是想,快快打走敌人,你才能快快地回家。”\\冰床滑进水淀中央,这里是没有边际的冰场。太阳从冰面上升起来,冲开了雾,形成一条红色的胡同,扑到这里来,照在冰床上。女人说:“爹活着的时候常说,日本人在这里,水生出去是打开一条活路,打开了这条路,我们就能活。你记着爹的话,不要为家里的事分心,好好打仗,我等你回来。”\\在杨柳树环绕的丁家坞村边,水生下了冰床。\\女人忍住泪,笑着说:“快去吧你!记着,好好打仗,快回来,我们等着你的胜利消息。”\\ \rightline{一九四六年河间}
\songti





\subsection{『视角』之感官``请鉴赏划线句的表现力''}
此类题目灵活多变,需要灵活处理.主要需要注意\\
\df{内容}\nm 修辞,分析,情感\\
\df{结构}



\subsection{『视角』之人称``本文人称的表达效果是?''}
\df{第一人称}\nm 增强带入感,引发直观真切的体验. \\
\df{第一人称儿童视角}\nm 天真细致,引发成人后的反思.\\
\df{第二人称}\nm 跳出个人视角, 隐含了与读者的对话,拉近与读者的距离,产生与读者的共情.\\
\df{第三人称}\nm 增强叙述说理的客观性,但也令读者感到疏远.
\large
\subsubsection{例题}
赏析以下选段:
\\\nm \fangsong
\\大堰河, 为了生活, 
\\在她流尽了她的乳汁之后, 
\\她就开始用抱过我的两臂劳动了; 
\\她含着笑,洗着我们的衣服, 
\\她含着笑,提着菜篮到村边的结冰的池塘去,
\\她含着笑,切着冰屑悉索的萝卜, 
\\她含着笑,用手掏着猪吃的麦糟, 
\\她含着笑,扇着炖肉的炉子的火, 
\\她含着笑,背了团箕到广场上去, 晒好那些大豆和小麦,
\\大堰河,为了生活, 
\\在她流尽了她的乳液之后, 
\\她就用抱过我的两臂,劳动了。
\large\songti
\\\texttt{本段用了第三人称,描写了大堰河为我辛苦劳作的画面,体现出了作者与大堰河之间地主儿子和贫苦农妇间的身份差异与厚障壁,反映了二人之间情感逐渐的疏远.}
\subsubsection{例题} 
赏析以下选段: 
\\\nm \fangsong
\\大堰河,今天,你的乳儿是在狱里, 
\\写着一首呈给你的赞美诗, 
\\呈给你黄土下紫色的灵魂, 
\\呈给你拥抱过我的直伸着的手, 
\\呈给你吻过我的唇, 
\\呈给你泥黑的温柔的脸颜,
\\呈给你养育了我的乳房, 
\\呈给你的儿子们,我的兄弟们, 
\\呈给大地上一切的, 
\\我的大堰河般的保姆和她们的儿子, 
\\呈给爱我如爱她自己的儿子般的大堰河。
\large\songti
\\\texttt{本段运用第三人称,表达了作者对大堰河与千千万万大堰河般勤劳善良但又命运悲苦的普通农妇的歌颂与真诚同情.反映了作者的真实情感,拉近与读者的距离,令人感动.}

\subsection{『视角』之时间``随时间推移切换场景,赏析其妙''}
我们需要注意以下三点:\\
\df{环境}\\
\df{人物}\\
\df{场景}\\
\subsubsection{例题}
从``鸡叫三遍''到结束,小说随着时间推移切换场景,赏析其构思之妙.
\texttt{开头用``寒冷黎明''交代水生父亲之死 ,体现生活艰难;女人快速划冰床送水生上战场,体现了她的识时务与支持抗战;女人对水生的殷勤寄语更是反映了希望战争快速胜利的希望;最后"太阳升起""形成了红色的胡同"象征着通向胜利之路.}
\nm \fangsong
\\\centerline{原文见上}
\\鸡叫三遍,女人就起来给水生做了饭吃。这是一个大雾天,地上堆满了霜雪。女人把孩子叫醒,穿得暖暖的,背上冰床,锁了梢门,送丈夫上路。出了村,她要丈夫到爹的坟上去看看。水生说以后回来再去,女人坚持要去。她说:\\“爹叫你出去打仗了,是他一个老人家照顾了全家。这是什么日子呀?整天价东逃西窜。你不在家,爹对我们娘俩的照顾,只怕一差二错,对不起在外抗日的儿子。夜里一有风声,他就把我们叫醒。他老人家背着孩子逃跑,累的痰喘咳嗽。这些个担惊受怕的日子,把他老人家累死。还有那年大饥荒……”
\\在河边,他们上了冰床。水生坐上去,抱着孩子,用大衣给她包好脚。女人站在床子后尾,撑起了竿。女人是撑冰床的好手,她逗着孩子说:“看你爹没出息,当了八年八路军,还得叫我撑冰床子送他!”\\她轻轻地跳上冰床子后尾,像一只雨后的蜻蜓爬上草叶。轻轻用竿子向后一点,冰床子前进了。大雾笼罩着水淀,只有眼前几丈远的冰道可以望见。河两岸残留的芦苇上的霜花飒飒飘落,衣服上立时变成银白色。她用一块长的黑布紧紧把头发包住,脸冻得通红,嘴里却冒着热气。她连撑几竿,然后直起身子来,向水生一笑。小小的冰床像离开了强弩的箭,摧起的冰屑,在它前面打起团团的旋花。前面有一条窄窄的水沟,水在冰缝里汹汹地流,她只说了一声“小心”,两脚轻轻地一用劲,冰床就像受了惊的小蛇一样,抬起头来,窜过去了。\\水生提醒她说:“你慢一些,疯了?”女人擦一擦脸上的冰雪和汗,笑着说: “同志!我送你到战场上去呀,你倒说慢一些!”\\“擦破了鼻子就不闹了。”“不会。这是从小玩熟了的东西,今天更不会。在这八年里面,你知道我用这床子,送过多少次八路军?”\\冰床在霜雾里飞行。“你把我送到丁家坞,”水生说,“到那里,我就可以找到队伍了。”\\女人呆望着丈夫。停了一会,才说:“你知道,我现在心里很乱。八年才见到你,你只在家呆了不到多半夜的工夫。我为什么撑的这么块?为什么着急把你送到战场上去?我是想,快快打走敌人,你才能快快地回家。”\\冰床滑进水淀中央,这里是没有边际的冰场。太阳从冰面上升起来,冲开了雾,形成一条红色的胡同,扑到这里来,照在冰床上。女人说:“爹活着的时候常说,日本人在这里,水生出去是打开一条活路,打开了这条路,我们就能活。你记着爹的话,不要为家里的事分心,好好打仗,我等你回来。”\\在杨柳树环绕的丁家坞村边,水生下了冰床。\\女人忍住泪,笑着说:“快去吧你!记着,好好打仗,快回来,我们等着你的胜利消息。”
\songti


\subsection{『形象』``赏析该描写在刻画形象上的妙处''}
\large 对于提问赏析描写在刻画形象上的妙处的题目,要注意以下几个点:\par
\df{拆分提干}\nm 分析哪个词对应了什么形象; \\
\df{立体地塑造了...}.
\subsubsection{例题}
\large 
孙犁的文字,"寄至味于淡薄".请以水生夫妻炕头对话为例对此加以赏析.\\
\texttt{水生以事业以家国大事为重,而在外因为吃鱼这一日常小事想起家中的日常生活,想起妻子,符合战士身份(1);女人想起水生,因为她生活艰辛,牵挂家中的顶梁柱在外打仗生死未卜,因此她无论白天夜晚都会想念他,情深义重,符合乡村女子的个性身份和生活(1);而女人听闻丈夫要走,虽然不舍但仍然送他,体现她深明大义(从另一个角度分析),因此人物形象丰满立体(1);语言平淡但体现出夫妻之间情深义重以及以国家为重的深厚情感(1)。}
\nm \fangsong 
\\\centerline{原文见上}
\\女人歪在炕上,笑着问:``说真的,这八九年,你想起过我吗?'' ``想过。''``怎么想法?''她逼着问
\\``临过平汉路的那天夜里,我宿在一家小店,小店里有个鱼贩子是咱们乡亲。我买了一包小鱼下饭,吃着那鱼,就想起了你。''
\\ ``胡说。还有吗?''``没有了。你知道我是出门打仗去了,不是专门想你去了。''
\\ ``我们可常常想你,黑夜白日。''她突然支着身子坐起来,问:``你能在家住几天?''
\\ ``就这一晚上。我是请假绕道来看你的。''``为什么不早些说?''``还没顾着啊!''
\\ 女人呆了。她低下头去,无力地仄在炕上。过了半天,她说:``那么就赶快休息吧,明天我撑着冰床子去送你。''〔乙〕
\songti \large

\subsection{『虚实』修辞之虚实}
虚实结合是一种常用的比喻手法.

\subsubsection{例题}
本段有何表达效果?
\nm\fangsong
\\寻梦?撑一支长篙,
\\向青草更青处漫溯;
\\满载一船星辉,
\\在星辉斑斓里放歌。
\large\songti
\\\texttt{本段虚实结合,描写长篙青草于满船星辉的场面,营造出梦幻般的氛围,表达了诗人心中自由自在的喜悦之情,将诗歌推向高潮.}


\newpage

\section{说理与议论}

\subsection{『举例』``请分析举例论证的效果''}
在这类题目中, 答题时必须包括\par 
\df{举了什么例子}, \par 
\df{例子阐述了什么观点}以及\par
\df{这样写有什么表达效果}\nm 典型事例,具有典型性.
\subsubsection{例题}
\large \songti
分析第一段以颜回为例说理的作用(3分).\\
\texttt{本段例举了颜回虽屈居于陋巷, 无施于事, 无见于言, 却不妨碍众人的推尊与后人的视之为圣人(1), 为前文只要修好身就能成为圣人的观点做了突出强调(1), 运用典型事例, 具有典型性(1).}
\nm \fangsong \par 草木鸟兽之为物,众人之为人,其为生虽异,而为死则同,一归于腐坏澌尽泯灭而已。而众人之中,有圣贤者,固亦生且死于其间,而独异于草木鸟兽众人者,虽死而不朽,逾远而弥存也。其所以为圣贤者,修之于身,施之于事,见之于言,是三者所以能不朽而存也。修于身者,无所不获;施于事者,有得有不得焉;其见于言者,则又有能有不能也。施于事矣,不见于言可也。自诗书史记所传,其人岂必皆能言之士哉?修于身矣,而不施于事,不见于言,亦可也。孔子弟子,有能政事者矣,有能言语者矣。若颜回者,在陋巷曲肱饥卧而已,其群居则默然终日如愚人。然自当时群弟子皆推尊之,以为不敢望而及。而后世更百千岁,亦未有能及之者。其不朽而存者,固不待施于事,况于言乎?\par 予读班固艺文志,唐四库书目,见其所列,自三代秦汉以来,著书之士,多者至百余篇,少者犹三、四十篇,其人不可胜数;而散亡磨灭,百不一、二存焉。予窃悲其人,文章丽矣,言语工矣,无异草木荣华之飘风,鸟兽好音之过耳也。方其用心与力之劳,亦何异众人之汲汲营营? 而忽然以死者,虽有迟有速,而卒与三者同归于泯灭,夫言之不可恃也盖如此。今之学者,莫不慕古圣贤之不朽,而勤一世以尽心于文字间者,皆可悲也!\par 东阳徐生,少从予学,为文章,稍稍见称于人。既去,而与群士试于礼部,得高第,由是知名。其文辞日进,如水涌而山出。予欲摧其盛气而勉其思也,故于其归,告以是言。然予固亦喜为文辞者,亦因以自警焉。
\songti


\subsection{『类比』``类比说理的特色是?''}
对于考察类比说理的题目需注意\par
\df{类比的本体和喻体}\par
\df{表达效果}\nm (抽象概念具象化)(生动形象)\par
\df{与文章主旨的关联}


\subsection{『说服』``哪个论证更有说服力?"}
\large 对于提问说服力的题目,要注意以下若干点:\par
\df{角度全面}\\
\df{语言严谨}\\
\df{论证手法}\\
\df{说理方法}\nm 例证,典型性\\
\df{因果}\nm 逻辑是否能自洽,核心概念内涵外延是否对应逻辑链\\
\df{修辞手法}\nm 例如:排比\\
\df{情景}\\
\df{角度}\\
如果原文有比较,答题过程中也应该注意比较.
\subsubsection{例题} 
\large \songti 第三段林劝后主夺回淮南诸州的话很有说服力,请分析其原因.\\ 
\texttt{先分析双方的形式:宋淮南各州兵力薄弱,疲乏,而我有思旧之民.指出此为恢复故境的可乘之机;其次具体详细地陈述了自己的计划:如何举兵,举兵之时向外界报告其为外叛;最后预设了起兵之后成与不成的不同应对,以解除后主的后顾之忧.}\\
\blk
\nm \fangsong 林仁肇,建阳仁,事闽味裨将,沉毅果敢,文身为虎,军中惟之林虎子,\\闽亡,久不见用。会州侵淮南,元宗拔为将,时周人正阳浮桥初城,扼援师道路,仁肇率敢死士千人,以舟实薪刍,乘风举火焚桥,周将张永德来争,会风回,火不得施,我兵少却,永德鼓噪乘之,遂败,仁肇独骑一马为殿,永德引弓射之,屡将中,仁肇辄格去,永德惊曰:「此壮士,不可逼也。」遂舍之而还,\\开宝中,密言于后主曰:「宋淮南诸州,戍守单弱,而连年出兵,灭蜀,平荆湖,今又取岭表,往返数千里,师旅罢弊,此在兵家为有可乘之势,请假臣兵数万,出寿春,渡淮,据正阳,因思旧之民以复故境,彼纵来援,吾形势已固,必不得志,兵起之日,请以臣举兵外叛闻,事成,国家衅其利,不成,族臣家,明陛下不预谋,后主惧不敢从,\\时皇甫继勋朱全贇掌兵柄,忌仁肇雄略,谋有以中之,会朝贡使自京师回,使使言仁肇密通中朝,见其画像于禁中,且已为筑大第,以待其至。后主方任继勋等惑其言使仁持鸩往毒之,俄卒。\\初,仁肇尤味陈乔所知,至时,乔叹曰:国势如此,而杀忠臣,吾不知所税驾也,然不能白其诬,仁肇卒,逾年,后主遂见讨,又逾年,国为墟矣。\\
\rightline{选自陆游<南唐书>列传第十一,有删改}
\songti


\newpage

\section{结构,思想,评价与概括}

\subsection{『构思』``请从构思角度赏析''}
对于提问构思的题目尤其要注意情感线索.例如,在文章<良宵>中,"鹅"这一物象贯穿全文,是一个十分重要线索.对于"鹅"的分析要注意其出现的段落与情感线索.我们需要考虑它的\textbf{来历},\textbf{经历},与它的\textbf{得失是如何推动情节发展的}.:\par
\df{结构}\par
\df{主旨}\par
\df{为什么能突出主旨}\\
\subsubsection{例题}
作品围绕``窥看自己的心魂''与自我对话,请从构思角度对此作赏析.\\
\texttt{本文首先提出心中难解的关于生死和写作的三个问题,并对自我思考过程展开反思.从尝试写作的试探,到沉迷写作的焦虑,最后到在对写作的坚持中获得与自己和解与释然,有深度.}
\subsubsection{例题}
小说中的"鹅"在全文构思中有重要作用,请加以赏析.\\
\texttt{鹅本是被人抛弃,由老太太捡来的,陪伴了她十三年,是她的精神寄托;鹅是老太太和孩子的矛盾冲突点(行文线索);鹅被孩子偷走,杀死,直接引发老太太和孩子的冲突,后老太太病倒,引出孩子给她做饭,推动了情节发展.为后文老太太消除误会照顾孩子做铺垫,突出老太太善良仁慈,抒发作者对善行义事的赞美.}
\subsubsection{例题}
本文重点写村民就医,而二三两段描绘集市场景,这种安排体现了作者的巧妙构思,请加以赏析.\\
\texttt{集市的场景让村民们去医院场景,多角度营造了充满活力和谐自然的乡村风貌,新颖独特.反映了到处充满生活气息的乡村.}
\subsubsection{例题}
文章标题为``风吹云动'',而文中较多笔墨写到雨,请从构思角度加以赏析.\\
\texttt{风吹云动,云除了云卷云舒飘动外,更呈现雨的动态身心状态生活.而雨是云所变,是云的另一种形态,也是具有生机的事物,使大地得以新生,也象征了人生的去旧迎新,引出后文中对人生意义的思考:云有淡泊有猛烈,而许多人的人生只有一种,难以拥有淡泊卷舒的境界,深化了主旨.}
\nm \fangsong 
\\\centerline{我与地坛}\\\centerline{史铁生}
\\设若有一位园神,他一定早已注意到了,这么多年我在这园里坐着,有时候是轻松快乐的,有时候是沉郁苦罔的,有时候优哉游哉,有时候栖惶落寞,有时候平静而且自信,有时候又软弱,又迷茫。其实总共只有三个问题交替着来骚扰我,来陪伴我。第一个是要不要去死?第二个是为什么活?第三个,我干嘛要写作? 现在让我看看,它们迄今都是怎样编织在一起的吧.
\\你说,你看穿了死是一件无需乎着急去做的事,是一件无论怎样耽搁也不会错过的事,便决定活下去试试?是的,至少这是很关健的因素。为什么要活下去试试呢?好像仅仅是因为不甘心,机会难得,不试白不试,腿反正是完了,一切仿佛都要完了,但死神很守信用,试一试不会额外再有什么损失。说不定倒有额外的好处呢是不是?我说过,这一来我轻松多了,自由多了。为什么要写作呢? 作家是两个被人看重的字,这谁都知道。为了让那个躲在园子深处坐轮椅的人,有朝一日在别人眼里也稍微有点光彩,在众人眼里也能有个位置,哪怕那时再去死呢也就多少说得过去了,开始的时候就是这样想,这不用保密,这些现在不用保密了。 
\\我带着本子和笔,到园中找一个最不为人打扰的角落,偷偷地写。那个爱唱歌的小伙子在不远的地方一直唱。要是有人走过来,我就把本子合上把笔叼在嘴里。我怕写不成反落得尴尬。我很要面子。可是你写成了,而且发表了。人家说我写的还不坏,他们甚至说:真没想到你写得这么好。我心说你们没想到的事还多着呢。我确实有整整一宿高兴得没合眼。我很想让那个唱歌的小伙子知道,因为他的歌也毕竟是唱得不错。我告诉我的长跑家朋友的时候,那个中年女工程师正优雅地在园中穿行;长跑家很激动,他说好吧,我玩命跑。你玩命写。
\\这一来你中了魔了,整天都在想哪一件事可以写,哪一个人可以让你写成小说。是中了魔了,我走到哪儿想到哪儿,在人山人海里只寻找小说,要是有一种小说试剂就好了,见人就滴两滴看他是不是一篇小说,要是有一种小说显影液就好了,把它泼满全世界看看都是哪儿有小说,中了魔了,那时我完全是为了写作活着。结果你又发表了几篇,并且出了一点小名,可这时你越来越感到恐慌。我忽然觉得自己活得像个人质,刚刚有点像个人了却又过了头,像个人质,被一个什么阴谋抓了来当人质,不走哪天被处决,不定哪天就完蛋。你担心要不了多久你就会文思枯竭,那样你就又完了。凭什么我总能写出小说来呢?凭什么那些适合作小说的生活素材就总能送到一个截瘫者跟前来呢?人家满世界跑都有枯竭的危险,而我坐在这园子里凭什么可以一篇接一篇地写呢?
\\你又想到死了。我想见好就收吧。当一名人质实在是太累了太紧张了,太朝不保夕了。我为写作而活下来,要是写作到底不是我应该干的事,我想我再活下去是不是太骨傻气了?你这么想着你却还在绞尽脑汁地想写。我好歹又拧出点水来,从一条快要晒干的毛巾上。恐慌日甚一日,随时可能完蛋的感觉比完蛋本身可怕多了,所谓不怕贼偷就怕贼惦记,我想人不如死了好,不如不出生的好,不如压根儿没有这个世界的好。可你并没有去死。我又想到那是一件不必着急的事。可是不必着急的事并不证明是一件必要拖延的事呀?你总是决定活下来,这说明什么?是的,我还是想活。
\\人为什么活着?因为人想活着,说到底是这么回事,人真正的名字叫作:欲望。可我不怕死,有时候我真的不怕死。有时候, ---说对了。不怕死和想去死是两回事,有时候不怕死的人是有的,一生下来就不怕死的人是没有的。我有时候倒是伯活。可是怕活不等于不想活呀?可我为什么还想活呢?因为你还想得到点什么、你觉得你还是可以得到点什么的,比如说爱情,比如说,价值之类,人真正的名字叫欲望。这不 对吗? 我不该得到点什么吗?没说不该。可我为什么活得恐慌,就像个人质?后来你明自了,你明白你错了,活着不是为了写作,而写作是为了活着。你明自了这一点是在一个挺滑稽的时刻。那天你又说你不如死了好,你的一个朋友劝你:你不能死,你还得写呢,还有好多好作品等着你去写呢。这时候你忽然明白了,你说:只是因为我活着,我才不得不写作。或者说只是因为你还想活下去,你才不得不写作。是的,这样说过之后我竟然不那么恐慌了。就像你看穿了死之后所得的那份轻松?一个人质报复一场阴谋的最有效的办法是把自己杀死。我看出我得先把我杀死在市场上,那样我就不用参加抢购题材的风潮了。你还写吗?还写。你真的不得不写吗?人都忍不住要为生存找一些牢靠的理由。你不担心你会枯竭了?我不知道,不过我想,活着的问题在死前是完不了的。
\\这下好了,您不再恐谎了不再是个人质了,您自由了。算了吧你,我怎么可能自由呢?别忘了人真正的名字是:欲望。所以您得知道,消灭恐慌的最有效的办法就是消灭欲望。可是我还知道,消灭人性的最有效的办法也是消灭欲望。那么,是消灭欲望同时也消灭恐慌 呢?还是保留欲望同时也保留人生?我在这园子里坐着,我听见园神告诉我,每一个有激情的演员都难免是一个人质。每一个懂得欣赏的观众都巧妙地粉碎了一场阴谋。每一个乏味的演员都是因为他老以为这戏剧与自己无关。 每一个倒霉的观众都是因为他总是坐得离舞合太近了。
\\我在这园子里坐着,园神成年累月地对我说:孩子,这不是别的,这是你的罪孽和福扯。
\blkx
\\\centerline{良宵}\\  \centerline{张楚}\\
她刚搬到麻湾时,村人并未觉得有何异样。这只是位干净的老太太,衣着朴素,脸上一水褶子,梳了低低的发髻,站在樱桃树下,束手束脚,竟有几分与年岁不相称的羞怯。隔壁妇人偶来瞅几眼,闲聊几句,才晓得是村里王静生的远房姨妈,想起要到乡下住上段时日,这才劳烦外甥在村西租了三间瓦房。行李也不甚多,几床被褥,一只泛黄的皮箱。随行只有一只白鹅。
\\好事的村妇们,借串门的名义在炕沿上东拉西扯。可这老太太,安静得像一只猫,也不插嘴。问她儿女几个?她说,两儿一女。问她老伴是否健在?她说,去世二十多年了。闲妇们渐渐没了兴致,不怎么来往。
\\那天从村西的土岗下过,见一孩子在前边跑,一帮孩子在身后追。那孩子蹽得比野兔子快,转眼就从她身边刮过,直刮到那岗上。那帮孩子呢,也就不再追,只在岗下骂个一通,才怏怏散去。老太太斜眼见那土岗上隐约探出个圆头,小心逡巡着岗下。见老太太望他,竟俯身捡起块土坷垃不偏不倚扔她额头上。老太太摸了摸额头,朝那岗上望去,孩子就不见了。
\\午后,老太太喝了碗稀饭,猫进被窝,看电视。过堂屋传来电饭锅被揭开的滋啦声,饭菜入嗓猛然吞咽的咕咚声……她蹑手蹑脚踱到庭院,见岗上那个孩子在往外翻墙。老太太默然看了片刻回了房。
\\翌日出门,买了冷鲜肉,切姜剥蒜,配了红椒、桂圆、八角、茴香,用高压锅将肉焖了。肉香四处散了开去,老太太眯眼打起盹儿来。等睁开眼,天已大黑,去过堂屋看炖的肉,明显是吃剩的。老太太竟有些隐隐的得意,方沉沉睡去。
\\次日早起,坐到屋檐下晒太阳,晒着晒着有些恶心,吞了几粒药片,倒头睡起来。醒来时太阳已爬上屋檐,却发现老鹅没了。
\\这老鹅,跟了她十三年,从小区门口捡的。小小一团鹅黄,谁承想竟长成偌大一只呢?儿女们是极少来的,通常只有她和它。想说话了和它唠叨两句,生气了就踹它两脚,它不记仇,依旧影子似的随着她,腻着她。
\\老太太在院子四周搜寻一番,仍没得踪迹。猛然想起那孩子,心就咯噔了一下。
\\那晚,她早早在过堂屋候了。果不其然孩子来了。当他在灶台上翻寻时,她一把就攥了他胳膊,问道:“是不是把鹅偷走了?”孩子点点头。她想也没想就在他后脑勺儿扇了一巴掌。“是不是把鹅给吃了?”孩子又是点点头。顺势拎了把刷锅的炊具,捋起他衣袖就抽打起来。抽着抽着便瞧得他胳膊上全是银元大小的红斑,一圈连一圈,看得心里麻麻幽幽,索性撒了他,一屁股坐在灶台上,默默盯了他半晌,摆摆手说:“你走吧,走吧。以后不要再来了。”孩子一愣,没有动,只嘟囔道:“我奶奶死了……我杀了它祭祀……”老太太不再搭理他,转身回了屋,和衣躺下。
\\一躺就是两天。再次睁开眼,屋里灯怎么就亮了。炕沿上摆着副碗筷,碗里尚冒着热气,是碗疙瘩汤。香油花浮着,白鸡蛋卧着。老太太心里热了下,吸溜起来。还好,病隔了一夜就痊愈了。
\\那天晚上,老太太喝完了汤,耳畔便传来谁家的收音机正在唱《春闺梦》,是张氏与丈夫王恢互诉衷肠那一场。听着听着,她不禁轻声唱将起来:
\\去时陌上花如锦,今日楼头柳又青!可怜侬在深闺等,海棠开日到如今。门环偶响疑投信,市语微哗虑变生。因何一去无音信?不管我家中肠断的人。
\\“咕咚”一声闷响,她才猛然梦醒,身子打个激灵,朝墙边看去,那孩子从墙头跌了下来。
\\“我……我……”男孩诺诺道,“我只是来瞧瞧,你的病好了没有。那天晚上,你的头比开水还热……”老太太领男孩进屋,给他热了排骨和米饭。
\\随后几日,男孩都过来共进晚餐。孩子通常只闷了头扒饭,很少动筷子搛菜。吃一阵偶然抬头,老太太便往他碗里搛一箸菜,孩子也搛了肉丁或腊肠,犹犹豫豫着往老太太碗里塞。老太太就笑。
\\当日晌午刚过,王静生就来了。王静生说,关于她跟孩子的事,他听别人说了。孩子爸妈、爷爷早死了,奶奶前几天也死了。孩子的病是父母遗传的艾滋病。那晚,老太太做好了饭菜,孩子却没来。
\\儿子第二天到了麻湾。老太太正在炕上收拾皮箱,儿子说:“哎,我真是白着急了,原来你已经准备回去了啊?这个礼拜日就是你寿日,香港的李老板做了你一辈子的戏迷,专门从香港飞来给你庆祝,光赞助费就掏二十万。饭店呢,就定在凯撒大酒店,省电视台要全程录像呢。”
\\老太太看他一眼,抽出皮箱拉杆,拍了拍儿子的肩,就朝土岗走去。儿子一见,蹙着眉喊:“妈!出租车在村东呢!”老太太大抵聋了,只顾弯着脊背拉着皮箱朝前走。儿子小跑着过去,在母亲身后边走边絮叨:“不瞒你说,赞助费说是二十万,其实给了五十万!不就听你唱两句《春闺梦》和《锁麟囊》?人家拿你当宝,傲气值几个钱呢?”
\\老太太径直走到了岗下,伸手擦了擦汗,将皮箱扔在土岗那厢,朝坡走去。这条坡不长,但是陡。老太太弯下腰身,晃晃悠悠往上爬,当眼前蓦然出现一只瘦骨嶙峋的小手时,她不禁抬起脖子瞅了瞅。当孩子的小手紧攥住她的掌心时,老太太身上忽就有了气力,手脚在瞬间就热了起来。有那么片刻,老太太确信双腿其实就踏在棉花般洁净干燥的云朵里,每向上微微跨一小步,就离天空和星辰近了半尺。
\\\rightline{(节选自《天涯》2012年第6期,有删改)}
\blkx
\nm \fangsong 
\\(文章见『比喻』一节) 
\blkx
\large \songti

\subsection{『排序』关于排序的题目}
\df{横线前后的文本}\\
\df{内容}\nm 话题\\
\df{逻辑}\nm 时间先后认知过程,逻辑与前文的呼应\\
\df{标点}\\
\df{试排}


\subsection{『判词』人物传记特色评价词}
我们经常能够在做人物传记题目时遇到诸如``坚正''之类的评价性词语, 进而要求分析人物. 这里要注意小词放大的技巧, 例如: \LARGE 坚 \large 持操守, \LARGE 正\large 直讲义. 注意要点\par
\df{小词放大}\par
\df{事例一概括与品格一}\par
\df{事例概括二与品格二}

\subsection{『事迹』``请概括人物事迹''}
\large
概括人物事迹时需要答道\par
\df{人物事迹一}\\
\df{人物事迹二}\\
\df{人物品格}\\
其中,特别需要注意一类伪装的题目
\df{侧面体现,烘托,说明}\\
如下:\\
\subsubsection{例题}
分析第五段的作用.\\
\texttt{最后一段先写陈桥在林仁肇被杀之后的悲叹,从侧面体现出林对朝廷的忠诚与被杀的宽屈;后写林死后不久南唐被灭的的事实,侧面体现出林对朝廷的重要.丰满了人物形象.对上文的补充叙述说明,是篇章结构的重要组成部分.蕴含了作者对林的同情.}\\
\nm \fangsong 
(文章见『说服』一节)


\subsection{『思情』``请分析全文的思想情感''}
分析时要特别注意题干中是说\df{具体分析}还是\df{大致概括}.
\subsubsectionmark{例题}
第16段最后两句``云是云,我们是我们.云不是云,我们不是我们.''意蕴丰富,请结合全文,对此加以评析.\\
\texttt{本句为作者的感慨,阐明了人生在世要学会从容淡定,淡雅淡泊的.对树立我们的生活态度很有启发.}
\nm \fangsong
\\(文章见『比喻』一节)
\large \songti

\subsection{『情变』``作者情感态度是怎么转变的?''}
对于情感题, 我们应注意\par
\df{全文逐段体会}\par
\df{找议论的句子}\nm (判断, 表推测)\par
\df{直接找情感关键词}

\subsubsection{例题}
\\小说第1段她仅仅是对他``点头致意'',第15段她却``直直地冲他微笑''.有人认为这一转变缺少铺垫,不切实际.对此你是否认同,说说你的看法. 
\\ \texttt{我不认同.奥一开始只是点头致意,表达了最基本的礼仪;随后因被窥视而感到不满;后来又因为里桑谨慎礼貌的行为留下了美好的印象,由此推测他是一位会忍耐,尊重的男人;最后她又因为他的骑士风度,从而想象不同情景下的他,产生海市蜃楼般的美好幻想.由此可见最后"直直地冲他微笑"有着充分的铺垫,是自然的心理变化的体现.(全文逐段体会)}
\subsubsection{例题}
\\第9段画线句运用动作细节和语言细节,刻画了老太太怎样的心理变化过程。请简要分析。
\\ \texttt{老太太首先"拾"炊具,"捋"衣袖,"抽"起来,后"心理麻幽幽的",并"撒",\\"默默"让他走,最后"不再搭理",体现了老太太由爱鹅被杀的愤怒到心疼男孩的不忍,再到心中矛盾纠结,最后无可奈何.}
\nm \fangsong 
\\\centerline{玻璃边界}
\\ \begin{center} 1 \end{center} \\当利桑德罗与奥德丽目光相遇时,她点头致意,就像出于礼貌问候一位餐厅服务生,比问候公寓楼的门卫还要少一分热情……利桑德罗已经擦净了第一扇玻璃窗,正是奥德丽办公室的那扇,随着他慢慢除去灰尘形成的薄膜,她逐渐显现,起初遥远而朦胧,随后便一点点靠近,由于玻璃越来越清澈,她分毫未动却越来越近。就像调整相机的焦距,就像慢慢将她据为己有。\\玻璃的透明渐渐揭开她的面纱。办公室的灯光从身后照亮她的头部,为她栗色的头发笼罩上一层麦田般的柔美和动感,麦穗与如饰带般落于颈后的美丽金黄的麻花辫纠缠。光线聚集于后颈,当她将浅色的柔软辫子拨到一边时,后颈上的光照亮了从背部蜿蜒向上的每一根金黄的绒毛,就像一把种子,即将在编织的发束里找到土壤。 \\她伏案工作着,对他无动于衷,对他人的工作无动于衷,那种卑躬屈膝的手工劳动,与她的截然不同。她正努力为百事可乐找一句精彩的、引人注目、朗朗上口的广告语。他感到不自在,担心自己手臂在玻璃上的挥舞使她分神。如果她抬起头,会是因为工人的打扰而一脸愤怒吗? \\如果她再次看他,会用什么样的眼神? \\``上帝啊,''她低声自言自语,``他们提醒过我会有工人来。但愿这个男人没有在观察我。我感觉在被窥视。我有点生气了,没法集中精力。'' \\ \begin{center} 2 \end{center} \\她抬起头,碰上了利桑德罗的目光。她想要发怒却没能做到。那张脸上有种东西令她吃了一惊。一开始,她没有注意他外表的细节。令她战栗的是别的东西。某种她几乎从未在男人身上见过的东西。她在自己的词汇表里拼命搜寻,作为一个以遣词造句为职业的人,她寻找着一个词汇,来形容这个办公室玻璃清洁工的态度和面孔。\\在一闪念间她找到了——礼貌。在这个男人的身上,在他的态度、距离感、点头的方式与奇妙地混杂着忧伤和欢乐的目光中的那种东西,是礼貌,难以置信地毫无粗俗的痕迹。 \\“这个男人,”她想,“他绝不会在凌晨两点钟歇斯底里地打电话请求原谅,他会忍耐。他会尊重我的孤独,我也会尊重他的。” \ “这个男人会为你做什么?”她马上自问。“他会请我吃晚饭,然后送我到家门口。他不会让我在夜里独自叫出租车离开。” \\正当她抬起目光、神不守舍之时,他在转瞬之间看见了她深邃的栗色大眼睛。他马上垂下目光,继续工作,但与此同时他想起她微笑了。这是他的想象吗?还是真的?他鼓起勇气望向她。女人对他微笑,非常短促,非常礼貌,然后就低下头继续工作。\\一个眼神足矣。他没想到会在一个美国女人的眼睛里看到忧郁。人们说她们都很坚强,很自信,很专业,很守时,不是说所有的墨西哥女人都软弱、摇摆、随性、拖沓,不,完全不是。问题在于,一个会在星期六来工作的女人可能是各种样子,也许温柔,也许亲热,但唯独不该是忧郁的。利桑德罗清楚地在这个女人的眼神里看到了忧郁。她怀着悲伤,也怀着渴望。她渴望着。这是她的眼神所诉说的:“我想要某种缺失的东西。” \\奥德丽不必要地把头压得很低,好躲进纸张文件中。这太荒唐了。她难道要爱上大街上第一个擦肩而过的男人,只为了和丈夫彻底分手,让他吸取教训,只是因为纯粹的反弹效应?那个工人很英俊,这是糟糕之处,他有着不寻常的几乎令人感到冒犯的骑士风度,不合时宜,仿佛在滥用他的弱势地位,但他同时有着明亮的眼睛,眼里流露出的悲伤和喜悦同样浓烈,他的皮肤呈橄榄色,鼻子短而尖,鼻翼翕动着,身形修长,卷发,年轻,胡须厚重。与他的丈夫迥然不同。他是——她又一次露出微笑——一个海市蜃楼。 \\ \begin{center} 3 \end{center} \\他也对她回以微笑。他的牙齿坚硬、洁白。利桑德罗想到,他极力避开了会使他在当他还是个有志青年时认识的人面前降低身份的工作。他曾接下一份在弗克拉尔餐馆做服务员的差事,当他不得不为一桌中学老同学服务时,场面十分难堪。所有人都事业有成,除了他。他令他们难堪,他们也令他难堪。他们不知道该怎么称呼他,对他说些什么。还记得和西蒙·玻利瓦尔队比赛的时候你进的那个球吗?这是他听到的最友善的话了,随之而来的是一阵令人尴尬的沉默。 \\他做不了办公室文员,从中学三年级起他就辍学了,不会速记法也不会用打字机。做出租车司机更不行。他嫉妒比他有钱的乘客,看不起比他穷的,墨西哥城混乱的交通令他发狂,让他火冒三丈,暴跳如雷,不停骂娘,变成各种自己不喜欢的样子……超市售货员,加油站雇员,他什么都做过,那是自然。不幸的是现在连这样的差事都没有了。他感恩能获得这份来美国的工作,感恩此刻正直视着他的这个女人的眼睛。 \\他并不知道,她不仅在看着他,也在想象他。她先他一步。她想象着各种情境下的他。她把铅笔放到牙齿间。他会喜欢什么体育运动?他看起来很强壮,很健美。电影,演员,他喜欢电影、歌剧、电视剧吗?他是那种会透露电影结局的人吗?当然不是。这一眼就看得出来。她直直地冲他微笑。她会忍不住给伴侣讲出电影、侦探小说的结尾,除了自己的故事,因为永远不知道会怎么结束。\\她头脑中的想法他也许已经猜到一二。他多想能坦率地告诉她,我不一样,不要相信外表,我不应该在做这些,这不是我,我不是你想象的那样。可他不能对玻璃说话,他只能爱上玻璃上的光,而光可以穿过玻璃,触碰她,光是他们共同所有。 \\(有删节)\\【注释】①选自短篇小说集《玻璃边界》。小说集通过九个短篇故事,生动形象地刻画了墨西与美国这对邻居在长达两百年的历史演变中形成的恩怨。作品出版于《北美自由贸易协定》正式生效后不久,此时美墨两国间的贸易壁垒有所弱化,然而两国人民间的交流依然存在一些难以解决的问题。②卡洛斯·富恩特斯(1928年11月11日-2012年5月15日):墨西哥作家,他的作品深刻刻画了墨西哥的历史和现实。由于对欧美文明的了解和对拉美落后现状的认识,比起其他的的拉美作家,富恩特斯作品中存在着更强烈的忧患意识。
\blkx \begin{center} 良宵(节选) \end{center}\\顺势拎了把刷锅的炊具,捋起他衣袖就抽打起来。抽着抽着便瞧得他胳膊上全是银元大小的红斑,一圈连一圈,看得心里麻麻幽幽,索性撒了他,一屁股坐在灶台上,默默盯了他半晌,摆摆手说:“你走吧,走吧。以后不要再来了。”孩子一愣,没有动,只嘟囔道:“我奶奶死了……我杀了它祭祀……”老太太不再搭理他,转身回了屋,和衣躺下。

\songti \nm 

\subsection{『意蕴』``请品读加点词并体会其中意蕴''}
对于这类题目,需要注意\\
\df{词语的本意}\\
\df{语境}\\
\df{这段话上下文的语境}
\subsubsection{例题}
\fangsong
然而阿Q虽然常优胜,却直待\CJKunderdot{蒙}赵太爷打他嘴巴之后,这才出了名。\\
\texttt{"蒙"意为承蒙. 挨打也像荣幸蒙恩,形象地刻画出阿Q与看客以丧失人格为代价而换来趋炎附势的变态心理,体现了长期以来的奴性人格.}
\nm \songti




\end{document}