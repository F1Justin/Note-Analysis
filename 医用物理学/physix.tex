\documentclass[12pt, a4paper, oneside]{ctexbook}
\usepackage{amsmath, amsthm, amssymb, bm, graphicx, hyperref, mathrsfs, parskip}

\title{{\Huge{\textbf{医用物理学}}}\\笔记}
\author{F1}
\date{\today}
\linespread{1.5}
\newtheorem{theorem}{定理}[section]
\newtheorem{definition}[theorem]{定义}
\newtheorem{lemma}[theorem]{引理}
\newtheorem{corollary}[theorem]{推论}
\newtheorem{example}[theorem]{例}
\newtheorem{proposition}[theorem]{命题}

\begin{document}

\maketitle

\pagenumbering{roman}
\setcounter{page}{1}

%\begin{center}
%    \Huge\textbf{前言}
%\end{center}~\

%这是笔记的前言部分. 
%~\\
%\begin{flushright}
%    \begin{tabular}{c}
%        Dylaaan\\
%        \today
%    \end{tabular}
%\end{flushright}

%\newpage
\pagenumbering{Roman}
\setcounter{page}{1}
\tableofcontents
\newpage
\setcounter{page}{1}
\pagenumbering{arabic}

\chapter{流体力学与血液流变学简介}
流体:没有固定形状,具有流动特征的物质.

\section{流体运动的描述}
\subsection{描述流体运动的方法}
\begin{itemize}
    \item 拉格朗日法:跟踪流体中的一点,描述其运动
    \item 欧拉法
\end{itemize}   

\subsection{速度场与定常流动}
\begin{itemize}
    \item 速度场:流体中每一点的速度,$v = (x, y, z, t)$
    \item 定常流动:速度场不随时间变化,$v = (x, y, z)$
\end{itemize}

\subsection{流线与流管}
\begin{itemize}
    \item 流线:流体运动方向的切线
    \item 流管:流线的集合(流线不会相交)
\end{itemize}

\section{理想流体与连续性方程}
\subsection{理想流体}
理想流体:无黏滞性,不可压缩。
\subsection{连续性方程}
理想流体作定常流动时,流管形状不变,且流管内流体不可压缩,故在任意时间间隔$\Delta t$内流经$S_1$与$S_2$的流体体积相等,即
\begin{align*}
    S_1v_1\Delta t = S_2v_2\Delta t \\
    S_1v_1 = S_2v_2 = Constant
\end{align*}

\section{伯努利方程}
\subsection{理想流体的伯努利方程}
\begin{equation*}
    \frac{1}{2} \rho v^2 + \rho gh + p = Constant
\end{equation*}
\subsection{伯努利方程的应用}
\subsubsection{水平管中压强与流速的关系}
对于水平管,伯努利方程简化为
\begin{equation*}
    \frac{1}{2} \rho v^2 + p = Constant
\end{equation*}
因此,压强与流速成反比。
文丘里流量计:对于水中1和2两截面处,有
\begin{align*}
    \frac{1}{2} \rho v_1^2 + p_1 &= \frac{1}{2} \rho v_2^2 + p_2 \\
    v_1S_1 &= v_2S_2
\end{align*}
联立上式得截面1处的流速为
\begin{equation*}
    v_1 = S_2 \sqrt{\frac{2(p_1 - p_2)}{\rho(S_1^2 - S_2^2)}}
\end{equation*}
又因为$p_1 - p_2 = \rho gh$,故管中流量为
\begin{equation*}
    Q = v_1S_1 = S_1 S_2 \sqrt{\frac{2\rho g h}{S_1^2 - S_2^2}}
\end{equation*}

\section{黏滞流体的运动}

\subsection{黏滞流体的伯努利方程}
流体克服黏滞力做功,机械能不断损失并转化为热能,故伯努利方程变为
\begin{equation*}
    \frac{1}{2} \rho v^2 + \rho gh + p = Constant - \frac{1}{2} \rho v^2_{\text{损}}
\end{equation*}
若流体在水平均匀管道中作定常流动
\begin{align*}
    \because h_1 = h_2,  v_1 = v_2 \\
    \therefore p_1 = p_2 + \Delta E, p_1 > p_2
\end{align*}
若流体在开放的等粗管道中作定常流动
\begin{align*}
    \because p_1 = p_2 = p_0,  v_1 = v_2 \\
    \therefore \rho gh_1 -\rho gh_2 =  \Delta E
\end{align*}

\section{物体在流体中的运动}
\subsection{物体在理想流体中的运动}
设$h_1 = h_2$,由伯努利方程得
\begin{equation*}
    \frac{1}{2} \rho v_1^2 + p_1 = \frac{1}{2} \rho v_2^2 + p_2
\end{equation*}
升力:物体获得相对流速方向垂直(横向)且向流速增大一侧的动力。

\subsection{物体在黏滞流体中的运动与斯托克斯定律}
图示小球所受力
\begin{align*}
    G = \frac{4}{3}\pi r^3 \rho_1 g, f_浮 = \frac{4}{3}\pi r^3 \rho_2 g
\end{align*}
固体在黏滞流体中作匀速运动还会受到黏滞阻力,若物体运动速度很小,则
\begin{align*}
    f_阻 = 6\pi r \eta v
\end{align*}
沉降速度(终极速度):
\begin{align}
    v_s = \frac{2(\rho_1 - \rho_2)}{9\eta}gr^2
\end{align}
用此公式可求得
\begin{itemize}
    \item 液体黏滞系数
    \item 球体半径
\end{itemize}


\section*{本章小结}
\begin{itemize}
    \item 连续性方程:流量$Q = Sv$, 连续性方程$Sv = Constant$
    \item 理想流体的伯努利方程:$\frac{1}{2} \rho v^2 + \rho gh + p = Constant$\\适用条件:\textbf{理想流体,定常流动,同一流管}\\伯努利方程应用说明:
    \begin{itemize}
        \item 正确地选取截面, 包含所求量
        \item 方程正确简化:对于\textbf{等粗管道},$p_1 + \rho g h_1 = p_2 + \rho g h_2$;对于\textbf{水平管道},$p_1 + \frac{1}{2}\rho v_1^2 = p_2 + \frac{1}{2}\rho v_2^2$
        \item 找出隐条件: 大管小孔, 大处$v$不计; 与空气接触, $p  = p_0$
    \end{itemize}
    \item 牛顿黏滞定律:$F = - \eta \Delta S \frac{\mathrm{d}v}{\mathrm{d}x}$,其中$\eta$为黏滞系数,单位为$Pa \cdot s$\\
    说明:\begin{enumerate}
        \item 黏度取决于流体性质
        \item 液体的黏度大于气体
        \item 与温度的关系:对液体$t\uparrow \eta \downarrow$,对气体$t\uparrow \eta \uparrow$
    \end{enumerate}
    \item 层流与湍流:\\雷诺数:$R_e = \frac{\rho v r}{\eta}$,$R_e > 1500$作湍流,$R_e < 1000$作层流,$1000 < R_e < 1500$不稳定,会互相转变
    \item 泊肃叶定律:$Q = \dfrac{\pi r^4 \Delta p}{8\eta l}$,其中$\Delta p$为压差,$l$为管长
    \item 黏滞流体的伯努利方程:$p_1 + \frac{1}{2}\rho v_1^2 + \rho g h_1 = p_2 + \frac{1}{2}\rho v_2^2 + \rho g h_2 + \Delta{E}$
    \item 斯托克斯定律:$f = 6\pi \eta v r$\\可推导出沉降速度:$v_s = \frac{2(\rho_1 - \rho_2)}{9\eta}gr^2$
\end{itemize}



\chapter{震动与波、声波、超声波}
研究对象:物体的周期性运动及其运动规律。\\
振动:周期性运动;波动:振动的传播。
\section{简谐运动}
\subsection{弹簧振子}
机械振动的原因:物体所受回复力和物体所具有的惯性。
回复力:始终指向平衡位置
\subsection{描述简谐运动的物理量}
\begin{itemize}
    \item 振幅:$A$:振动的幅度
    \item 角频率:$\omega = 2\pi f$:2$\pi$秒内往复振动的次数
    \item 相位:$\varphi = \omega t + \varphi_0$:
    \item 初相:$\varphi_0$:$t = 0$时刻的相位
    \item 周期:$T = \frac{1}{\nu} = \frac{2\pi}{\omega}$:振动一次所用时间
    \item 频率:$\nu = \frac{1}{T} = \frac{\omega}{2\pi}$:单位时间内振动的次数
\end{itemize}
\subsection{简谐运动的速度和加速度}
简谐运动表达式
\begin{align*}
    x = A\cos(\omega t + \varphi)
\end{align*}
简谐运动的速度
\begin{align*}
    v = \frac{\mathrm{d}x}{\mathrm{d}t} = -A\omega\sin(\omega t + \varphi)
\end{align*}
简谐运动的加速度
\begin{align*}
    a = \frac{\mathrm{d}v}{\mathrm{d}t} = -A\omega^2\cos(\omega t + \varphi)
\end{align*}
而$v_m = \omega A$,$v_m$称为速度幅
故简谐运动的加速度可表示为
\begin{align*}
    a = -\omega^2x
\end{align*}
对于弹簧系统,由牛顿第二定律
\begin{align*}
    F = ma = -m\omega^2x
\end{align*}
又胡克定律
\begin{align*}
    F = -kx
\end{align*}

\subsection{简谐运动的旋转矢量表示法}

\subsection{简谐运动的能量}
\begin{itemize}
    \item 振子势能:$E_p = \frac{1}{2}kx^2 = \frac{1}{2}kA^2\cos^2{(\omega t + \varphi)}$
    \item 振子动能:$E_k = \frac{1}{2}mv^2 = $
\end{itemize}

\section{简谐运动的合成}
\subsection{两个同方向同频率的简谐运动合成}
一个质点参与两个在同一直线上频率相同的简谐运动,其合运动仍为简谐运动,其振幅为两个简谐运动振幅的矢量和。
\subsection{两个同方向不同频率的简谐运动合成}









\end{document}