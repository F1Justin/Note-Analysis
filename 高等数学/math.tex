\documentclass[12pt, a4paper, oneside]{ctexbook}
\usepackage{amsmath, amsthm, amssymb, bm, graphicx, hyperref, mathrsfs}

\title{{\Huge{\textbf{高等数学}}}\\笔记}
\author{F1}
\date{\today}
\linespread{1.5}
\newtheorem{theorem}{定理}[section]
\newtheorem{definition}[theorem]{定义}
\newtheorem{lemma}[theorem]{引理}
\newtheorem{corollary}[theorem]{推论}
\newtheorem{example}[theorem]{例}
\newtheorem{proposition}[theorem]{命题}

\begin{document}

\maketitle

\pagenumbering{roman}
\setcounter{page}{1}

%\begin{center}
%    \Huge\textbf{前言}
%\end{center}~\

%这是笔记的前言部分. 
%~\\
%\begin{flushright}
%    \begin{tabular}{c}
%        Dylaaan\\
%        \today
%    \end{tabular}
%\end{flushright}

%\newpage
\pagenumbering{Roman}
\setcounter{page}{1}
\tableofcontents
\newpage
\setcounter{page}{1}
\pagenumbering{arabic}

\chapter{章节标题}

在这里可以输入笔记的内容. 

\section{数列的极限}

\subsection{数列极限的定义}
\begin{example}
    已知$x_n = \frac{n + (-1)^n}{n}$, 求$\lim\limits_{n \to \infty}x_n = 1$.\\
    解: $\forall \epsilon > 0$给定, 要寻找$N$, 使当$n > N$时, $|x_n - 1| < \epsilon$.\\
    $|x_n - 1| = |\frac{n + (-1)^n}{n} - 1| = \frac{1}{n}$,\\
    由于$\frac{1}{n} < \epsilon$, 所以$n > \frac{1}{\epsilon}$, 取$N = \frac{1}{\epsilon}$, 则当$n > N$时, $|x_n - 1| < \epsilon$.\\
    所以$\lim\limits_{n \to \infty}x_n = 1$.
\end{example}

\begin{example}
    s
\end{example}

\subsection{收敛数列的性质}
\begin{enumerate}
    \item 极限唯一性: 若$\lim\limits_{n \to \infty}x_n = A$, 则$A$唯一.
    \item 收敛数列必有界
    \item 收敛数列的保号性: 若$\lim\limits_{n \to \infty}x_n = A > 0$, 则$\exists N$, 当$n > N$时, $x_n > 0$.
    \item 收敛数列的
\end{enumerate}




\end{document}