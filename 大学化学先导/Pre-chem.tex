\documentclass[12pt, a4paper, oneside]{ctexbook}
\usepackage{amsmath, amsthm, amssymb, bm, graphicx, hyperref, mathrsfs}

\title{{\Huge{\textbf{大学化学先导}}}\\笔记}
\author{F1}
\date{\today}
\linespread{1.5}
\newtheorem{theorem}{定理}[section]
\newtheorem{definition}[theorem]{定义}
\newtheorem{lemma}[theorem]{引理}
\newtheorem{corollary}[theorem]{推论}
\newtheorem{example}[theorem]{例}
\newtheorem{proposition}[theorem]{命题}

\begin{document}

\maketitle

\pagenumbering{roman}
\setcounter{page}{1}

%\begin{center}
%    \Huge\textbf{前言}
%\end{center}~\

%这是笔记的前言部分. 
%~\\
%\begin{flushright}
%    \begin{tabular}{c}
%        Dylaaan\\
%        \today
%    \end{tabular}
%\end{flushright}

%\newpage
\pagenumbering{Roman}
\setcounter{page}{1}
\tableofcontents
\newpage
\setcounter{page}{1}
\pagenumbering{arabic}

\chapter{绪论}

\section{甜与分子结构}
从阿斯巴甜到纽甜

\section{顺反异构与物质稳定性}
不饱和脂肪酸的稳定性:天然不饱和脂肪酸多为顺式结构。反式脂肪酸熔点较高,较为稳定。

\section{物质熔点和结构间的关系}
\begin{itemize}
    \item 有机物的熔点与分子间的相互作用力有关,分子间的相互作用力越强,熔点越高。
    \item 分子间的相互作用力与分子间的距离有关,分子间的距离越近,分子间的相互作用力越强。
    \item 分子间的距离与分子的结构有关,分子结构越紧密,分子间的距离越近。
\end{itemize}
离子键和共价键并没有绝对的界限。




\section{小节标题}

 
\subsection{test}


\end{document}